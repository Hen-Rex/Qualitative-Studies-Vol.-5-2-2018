\label{paper5:references}
\begin{thebibliography}{99}
%%%%%%%%%%%%%%%%%%%%%%%%%%%%%%%%%%%%%%%%%%%%%%%%%%%%%%%%
%%%%%%%%%%%%%%% START 9_references.tex %%%%%%%%%%%%%%%%%
%%%%%%%%%%%%%%%%%%%%%%%%%%%%%%%%%%%%%%%%%%%%%%%%%%%%%%%%

\item Anderson, Ben and Colin McFarlane. 2011. “Assemblage and Geography.” AREA 43(2): 124-127.
\item Anderson, Ben, Matthew Kearnes, Colin McFarlane and Dan Swanton. 2012. “On Assemblages and Geography” Dialogues in Human Geography 2(2): 171-189
\item Bedford, Kate D. 2015. “Regulating Bingo: Lessons from the Bingo Halls.” Law \& Social Inquiry 40(2): 461-490.
\item Citizens Advice Bureau. 2006. ‘Pro Bono Legal Services and Citizens Advice Bureaux’, NDT briefing. London: Citizens Advice.
\item Citizens Advice. 2015. The value of the Citizens Advice service: our impact in 2014/15 [available at \url{https://www.citizensadvice.org.uk/Global/Public/Impact/Citizens\%20Advice_Impact\%20Report_2015_Digital.pdf}].
\item Citron, Judith. 1989. The Citizens Advice Bureaux: for the community by the community. London: Pluto Press.
\item Clarke, John and Janet Newman. 1997. \textit{The Managerial State: Power, politics and ideology in the remaking of social welfare}. London: Sage Publications.
\item Clarke, John, Dave Bainton, Noemi Lendvai and Paul Stubbs. 2015. \textit{Making Policy Move: towards a politics of translation and assemblage}. Bristol: Policy Press.
\item Clarke, J, McDermont, MA \& Newman, J 2010. Delivering Choice and Administering Justice: Contested Logics of Public Services. in M. Adler (ed.), \textit{Administrative Justice in Context}. Hart, pp. 25--46.
\item DeLanda, M. 2006. A New Philosophy of Society: Assemblage Theory and Social Complexity. London: Continuum.
\item DeLanda, M. 2016. Assemblage Theory. Edinburgh: Edinburgh University Press.
\item Deleuze, G. and F. Guattari. 1988. A Thousand Plateaus, London: Continuum.
\item Dunn, Alison. 2008. “Demanding Service or Servicing Demand? Charities, Regulation and the Policy Process.” Modern Law Review 71(2): 247-70.
\item Food Foundation 2017 “UK and global malnutriotion: The new normal” [Report] [Available from \url{https://foodfoundation.org.uk/wp-content/uploads/2017/07/1-Briefing-Malnutrition\_v4.pdf}].
\item Forbess, A. and James, D.. Forthcoming. “Inventing intervention in the time of austerity.” Onati Socio-Legal Series.
\item Fraser, D. (1973). The Evolution of the British Welfare State: A History of Social Policy since the Industrial Revolution. Basingstoke: Palgrave MacMillan
\item Hacking, Ian. 1986. ‘Making up people’ in Reconstructing Individualism, edited by T.C. Heller. Stanford: Stanford University Press.
\item Jones, Rhys. 2010. “Learning beyond the state: the pedagogical spaces of the CAB service.” Citizenship Studies 14(6): 725-738.
\item Larner, Wendy and Higgins, Vaughan. 2017. “Conclusion: Awkward Assemblages” in Larner \& Higgins (eds.) Assembling Neoliberalism: Expertise, Practices, Subjects. New York: Palgrave Macmillan
\item Lewis, Jane. 2005. “New Labour's Approach to the Voluntary Sector: Independence and the Meaning of Partnership.” Social Policy and Society  4 (2): 121 – 131
\item Mayo, Marjorie, Gerald Koessl, Matthew Scott and Imogen Slater. 2015. Access to justice for disadvantaged communities, Bristol: Policy Press.
\item McDermont, M. (2010). \textit{Governing independence and expertise: the business of housing associations}. Oxford: Hart.
\item McFarlane, Colin. 2011. “The city as assemblage: dwelling and urban space.” Environment and Planning D 29: 649-671.
\item Mellaard, Arne and Toon van Meijl. 2016. “Doing policy: enacting a policy assemblage about domestic violence”. Critical Policy Studies [\url{http://dx.doi.org/10.1080/19460171.2016.1194766}].
\item Ministry of Justice. 2012. Digital Strategy [available at \url{http://open.justice.gov.uk/digital-strategy}].
\item Moore, Sarah and Alex Newbury. 2017. Legal Aid in Crisis: assessing the impact of reform. Bristol: Policy Press
\item Morrison, John. 2000. “The government – voluntary sector compacts: governance, governmentality and civil society.” Journal of Law and Society 27: 98-132.
\item Musick, Marc and John Wilson. 2008. Volunteers: A Social Profile. Bloomington, IN: Indiana University Press
\item Newman, Janet and John Clarke John. 2009. \textit{Publics, politics and power}. London: Sage Publications.
\item Ong, A. and S. Collier, eds. 2005. Global assemblages: technology, politics and ethics as anthropological problems.  Oxford: Blackwell Publishing
\item Osborne, Stephen P.  and Piers Waterston. 1994. “Defining contracts between the state and charitable organisations in national accounts: a perspective from the UK.” Voluntas 5(3): 291-300
\item Paton, R. 2003. Managing and Measuring Social Enterprises. London: Sage.
\item Power, Michael. 1997. The Audit Society: Rituals of verification. Oxford: Oxford University Press.
\item Rabinow, Paul. 2003. Anthropos Today: Reflections on Modern Equipment. New Jersey: Princeton University Press.
\item Randalls, Samuel. 2017. “Assembling Climate Expertise: Carbon Markets, Neoliberalism and Science” in Larner \& Higgins (eds) 2017 (ibid).
\item Rochester, Colin, Angela Ellis Paine and Steven Howlett. 2010. Volunteering and Society in the Twenty First Century, Basingstoke, UK: Palgrave MacMillan.
\item Shelter (2017) “Far from alone: Homelessness in Britain in 2017” [Report]. [Available from: \url{http://england.shelter.org.uk/__data/assets/pdf_file/0017/1440053/8112017_Far_From_Alone.pdf}].
\item StepChange (2017) “Stuck in the red: StepChange Debt Charity client stories of persistent overdraft debt” [Policy Brief]. [Available from \url{https://www.stepchange.org/Portals/0/documents/Reports/stuck-in-the-red-december-2017.pdf}].
\item Thompson, Piers and Williams, Robert (2014) “Taking Your Eyes Off the Objective: The Relationship Between Income Sources and Satisfaction with Achieving Objectives in the UK Third Sector.” Voluntas 25:109–137.
\item Wolch, J. (1989) “The Shadow State: Transformations in the Voluntary Sector” in Wolch, J. and Dear, M. (eds.) The Power of Geography: How Territory Shapes Social Life. London: Routledge

%##################################################################################################################################################################################################################################
\end{thebibliography}