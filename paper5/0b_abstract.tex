    \begin{adjustwidth}{20mm}{20mm}
\label{paper5:abstract}
    \bigskip
    \begin{otherlanguage}{english}
    {\small
%    \fadebreak
%%%%%%%%%%%%%%%%%%%%%%%%%%%%%%%%%%%%%%%%%%%%%%%%%%%%%%%%
%%%%%%%%%%%%%%%% START ABSTRACT.tex %%%%%%%%%%%%%%%%%%%%
%%%%%%%%%%%%%%%%%%%%%%%%%%%%%%%%%%%%%%%%%%%%%%%%%%%%%%%%

\noindent Voluntary sector advice agencies play, for many citizens in the UK, a key role in accessing and understanding public services. As such, whilst fiercely ‘independent’, their relationship to the welfare state is a complex and conflicted one. Presenting data from participant observation, interviews and focus groups with advisers and managers within the Citizens Advice Service, this paper explores this relationship by focusing on two particular areas of the service; the delivery of the service by volunteers, and the different funding streams that enable service to function. The paper draws upon assemblage theory, focusing as it does upon elements of an organisation in their ongoing practices and relationships; a processual approach that allows us to reflect upon the broader implications of our ethnographic data. Whilst this approach was motivated by our interest in how the Citizens Advice service endures, we conclude by reflecting upon the ‘fragile futures’ of advice in the context of aggressive budget cuts and the welfare reform agenda.

\smallskip
\noindent\rule{\linewidth}{1pt}

%%%%% KEYWORDS
%
\noindent{\bfseries Keywords:}\hspace*{0.75em}{% list keywords below
advice agencies;
assemblages;
regulation;
legal actors;
voluntary sector;
volunteers}.





    } % ends font family

%    \fadebreak

    \end{otherlanguage}

    \end{adjustwidth}