%%%%%%%%%%%%%%%%%%%%%%%%%%%%%%%%%%%%%%%%%%%%%%%%%%%%%%%%
%%%%%%%%%%%%%% START INTRODUCTION.tex %%%%%%%%%%%%%%%%%%
%%%%%%%%%%%%%%%%%%%%%%%%%%%%%%%%%%%%%%%%%%%%%%%%%%%%%%%%

%\chapter{Introduction}
\lettrine[lines=2]{\bfseries\color{black}F}{ew periods} in the history of the UK welfare state have seen such rapid change as the first years of the ‘coalition’ government in the UK, roughly those between 2010 and 2013. Introduced under the banner of ‘austerity’, these changes were, among others: the introduction of a more punitive welfare benefits system (Universal Credit) alongside a general ‘freeze’ on benefit entitlements, the (near) abolition of Civil Legal Aid (which had enabled citizens to challenge administrative decision-making, see Hynes, 2013), and far reaching cuts to  the funding of local government%
    \pagenote{In the United Kingdom, local government is referred to as the Local Authority, the ‘local council’ or simply ‘the Council’. In this paper we, and our interviewees, use the terms interchangeably.},
leading to drastic cuts in public services at the local level. At the time of writing, in the spring of 2018, the catastrophic effects of these changes are becoming starkly apparent: spiralling homelessness (Shelter, 2017), a household debt crisis (StepChange, 2017) and stark warnings of endemic child malnutrition in impoverished areas (Food Foundation, 2017).
\par
In such circumstances, voluntary sector advice agencies have become, for many in the UK, a vital safety net as they seek to manage the effects of these changes. They can enable access to benefits and services in a climate in which administrating agencies are guided primarily by punitive aims (Wacquant, 2009) – that is, to deny, rather than facilitate, access to financial or other supports. In these difficult times, legal protections become increasingly important; advice agencies can provide advice to citizens on the legal sphere regardless of their financial resources. The line dividing advice agencies from the welfare state, in its traditional interpretation as a set of institutions mitigating the effects of market forces (Fraser, 1973:1), is increasingly blurred.
\par
It was in this time of political and social turmoil that we conducted a programme of research investigating advice services, with a particular focus upon the largest advice provider in the UK, the Citizens Advice service.%
    \pagenote{\url{https://www.citizensadvice.org.uk}.}
The local Citizens Advice office or ‘bureau’%
    \pagenote{The word ‘bureau’, perceived by some as old-fashioned, was dropped in 2015. We use it in this paper as it continues to be used by both advisers and the public when referring to advice locations.}
is, for many, the only lifeline in dealing with multiple everyday problems and crises – from losing a job, mounting debts, threats of eviction or the demands of caring for elderly relatives or sick children. Staffed primarily by volunteers, bureaux have been faced with a surplus in demand spurred by the above described ‘austerity’ measures (Clarke, et al., 2015), coupled to increasingly insecure and precarious employment contracts, labour conditions and punitive immigration policies. The crisis for advice services comes from two directions: ever-more complex client problems requiring more intense and specialist support, and a resourcing crisis, as local authorities, who previously provided most funding for most bureaux, make ever-deeper budget cuts.
\par
In this paper we draw upon field notes from Samuel Kirwan’s participant observation as an adviser in a Citizens Advice Bureau, as well as interviews with advisers and managers within the Citizens Advice service, to describe how the service has dealt with, and been transformed by, these challenges. In doing so we aim to go beyond thinking which addresses such services solely in terms of their ‘front-line’ work. Here, we build upon our collective experience in working in, and researching, the voluntary and public sectors, to explore the ways in which the possibilities and limits of action rests not only upon the practice of advice, but also upon the variety of financial and regulatory pressures which shape how institutions deliver services to multiple publics.
\par
Over the course of our research, as we took part in and explored advice practice, interviewed managers on funding arrangements, regulatory mechanisms, and their organising of a voluntary workforce, we were intrigued by how an organisation composed of such heterogenous elements nevertheless cohered, holding together to provide ‘advice solutions’ in deeply challenging circumstances. It was through trying to understand this coherence that we came to work with assemblage-thinking as an analytical tool. This paper as such seeks to understand the changing role of advice work through the lens of the ‘advice assemblage’, using the framework of assemblage theory to understand how such organisations%
    \pagenote{While we seek to understand the broad provision of voluntary advice, we recognise that, as our research is focused upon the Citizens Advice Service, our references to advice work and the ‘advice assemblage’ are to some extent specific to this service. On a nationwide scale there are also specialist services also offering face-to-face advice , notably Shelter and AgeUK, and many local advice services in the towns and cities of the UK.}
can survive and continue to meet the needs of citizens as we enter a period of further cuts and hardship.
\par
As we describe, the ‘advice assemblage’ is composed of multiple components at different temporal and geographic scales (see also Jones, 2010). Rather than propose an exhaustive mapping of this assemblage however, we focus on two such components that are key, we argue, to understanding the changing relationship of the advice sector to the welfare state. These are: the voluntary workforce that provides frontline advice, and the funding arrangements that enable the delivery of advice.
\par
In Section 2 we first set out the research methods used to collect the data for this paper, then provide in Section 3 some background to the Citizens Advice service, describing its positioning in the voluntary sector in the UK and with respect to the welfare state. Section 4 sets out five key elements of assemblage thinking which we believe provide insights into understanding advice services. In Section 5 we draw upon our ethnographic data to explore how the voluntary workforce and funding arrangements within the advice assemblage, create, maintain, and potentially destabilise, the advice assemblage. We conclude, in Section 6, with our concerns for the ‘fragile futures’ of advice.

\chapter{Research Methods}
This paper draws upon research carried out between 2012 and 2016 as part of the ‘New sites of legal consciousness: a case study of UK advice agencies’ programme.%
    \pagenote{Funded by the European Research Council, ERC-2011-StG\_20101124, Project No. 284152.}
The paper presents a combination of ethnographic and interview material: participant observation from a Citizens Advice Bureau, audio diaries recorded by trainees on the advice training programme; and 32 interviews and three focus groups with advisers. The paper draws also upon interviews with managers and directors/CEOs of 12 local Citizens Advice offices, along with three focus groups with CAB managers, all carried out between 2012 and 2015 in local advice offices across the nation regions of the (not-so) United Kingdom.%
    \pagenote{The research received ethical approval from the University of Bristol Faculty of Social Sciences and Law.}
\par
While the interviews with advisers were strongly structured, addressing issues of advice practice and its relationship to law, the interviews with the directors/CEOs were semi-structured, designed to develop an understanding of: the practices of the local advice offices in relation to developing resources; the relationships maintained at a local level; and the influences (internal and external) these managers perceived to be impacting on their work.%
    \pagenote{In sum, data was collected through participant observation and diaries of 3 trainee advisers; and 42 interviews and 5 focus groups with advisers, managers, trainers and trainees at CAB in England and Scotland. A further set of interviews was conducted with 10 managers of local CAB. In order to maintain the anonymity of research participants all names given are pseudonyms.}
We analysed the data collectively, identifying emerging themes, discourses and imaginings that occurred across interviews and contexts, through which we sorted and organised the data set.
\par
The combination of methods used in the project was designed to blur the boundary between insider and outsider perspectives. While we remain researchers fascinated by the role and ongoing functioning of the Citizens Advice service, our use of participant observation, and incorporation of Author 2’s developing consciousness as an adviser in analysing and sorting the data, allows for an immanent account of this key ‘public’ service. Indeed, it was an approach that allowed for an attention to the embodied, technical and procedural elements, and their role in holding this service together, that forms the basis of this paper.
 
\chapter{Citizens Advice: Some Background}
Citizens Advice bureaux have existed in the UK since 1939, set up to meet the needs of citizens dislocated by wartime turmoil (Citron 1989).  Each local Citizens Advice Bureau is an independent charity run by a local board of trustees, a Director or Chief Executive Officer, paid staff and volunteers. They have become part of the fabric of local life, with municipal signage providing direction to the nearest office. Indeed, many people think that Citizens Advice is part of local government, as one adviser explains:
    \blockquote[Allanah, generalist adviser in a semi-urban bureau]{There’s…well there’s a number that seem to think we are…the fact that we are volunteer-based and a charity is missed.  A lot of people seem to think we are an office of the Jobcentre.}
\noindent The strength at the local level derives in part from being placed in a national network. Each local office is a member of the national umbrella organisation: Citizens Advice (CitA) is the national organisation for England, Wales and the North of Ireland,%
    \pagenote{However, the relationship in the North of Ireland is somewhat different. The NI Regional office runs its own national client database and have not been required to introduce Gateway as mandatory. Audit is carried out by CitA.}
Citizens Advice Scotland (CAS) for Scottish bureaux. The national organisations provide support and resources through, for example: training programmes for volunteers; a database system that records details of all clients who access Citizens Advice and what action was taken; periodic audit to set and maintain standards; and social policy work that translates the everyday advice problems dealt with at a local level into proposal for social policy change (McDermont, 2013). Membership of the national organisation places certain requirements on local offices, such as operating the database system and submitting to periodic audit.
\par
Due to the specific development of the Citizens Advice service alongside the UK post-war welfare state, there are no real functional equivalents in other European settings. In the European states we have surveyed an individual seeking, for example, benefits or employment advice would go to separate state agencies or charities for each of these areas. Services providing more ‘generalist’ advice, such as AWO and Diakonie in Germany, typically provide help only to their own members. Furthermore, it is the model of this advice being provided on a voluntary basis that makes Citizens Advice so unusual.
\par
Citizens Advice, like other key actors in the voluntary sector provision of public services, has undergone immense changes over the decades. Just as the politics and practices of the 1980s and 90s produced a shift in the state from ‘a regime dominated by bureau-professionalism to one dominated by managerialism’ (Clarke \& Newman, 1997, 140), so the voluntary sector also underwent transformation.  Since the 1980s successive governments have sought to use voluntary sector organisations as alternative providers of public services: in the 1980s and 90s Conservative governments used the tool of contracting-out, leading to unease and concern about the ‘independence’ of voluntary organisations (Lewis 2005, 121). In the first decade of the twentieth century, New Labour, in part responding to these concerns, attempted to reshape relations between state sector and voluntary sector organisations through the ‘Compact’ (see e.g. Morrison 2000). 
\par
Our interviews with managers of local Citizens Advice bureaux would suggest they may not have experienced these shifts in the earlier periods to quite the same intensity as others in the voluntary sector. For Citizens Advice offices there appears to have been a degree of insulation from the managerialism of the 90s perhaps resulting from a number of vectors: their relations with local government which saw them funded through grants which provided a contribution to the running of the organisation rather than requirements for the provision of specific services under contract; their use of volunteers as ‘labour’ in delivering advice; and the tying in of the local CAB with the national organisation through the brand which had long been understood as a source of advice, and a guarantee of quality. We speculate that the fourth vector insulating advice services from the pressures of state managerialism was the nature of the service CAB provided: the state in the UK has never sought to provide a generalist advice service and so had less knowledge or inclination to prescribe and regulate what the voluntary sector did. However, in the period of our research the political landscape has been dominated by the politics and policies of ‘austerity’; severe reductions in public spending, particularly affecting legal aid, welfare and housing benefits and local government funding, have had profound implications for Citizens Advice which has seen a dramatic rise in demand for advice alongside cuts and reshaping of funding (see McDermont, 2017).
\par
Amidst these changes, the Citizens Advice Bureau, as both a local office providing advice and information on the broad range of issues that comprise citizen-state relationship, and a complex organism composed of a variety of elements working at different scales (also see Jones 2010), has remained broadly unchanged. Our fieldwork with those who worked at the local level shone a light on how, in the ongoing practices of advice bureaux, particular practices, ideas and relationships held the disparate elements of the organisation together. Seeking to combine our ethnographic and interview work with advisers with these observations on the changing ‘complexity’ of the service led us to the theoretical tools offered by assemblage theory.

\chapter{Theoretical Tools: On Assemblage Thinking}
It was less the external view of the advice organisation, that is, its shape and form as viewed from outside its walls, than Samuel’s specific experience of training and advising and the emphasis in his diaries on the technical and procedural concerns, that led us initially to assemblage thinking.  It was clear from all our fieldwork that those ‘technical’ elements that came up in our discussions and focus groups – practices of audit, the terms of funding contracts, volunteer training delivery – could not be treated as neutral. Audit and auditors do not simply count and measure things; in requiring information about certain things and practices and for its presentation in prescribed formats, the process of audit changes organisations as they seek to conform and demonstrate excellence (e.g. Power 1997). Similarly, funders, (public, private and charitable) shape both organisational values and practices (e.g. McDermont, 2010 on Housing Associations). So, in seeking to understand local Citizens Advice, we started out with the assumption that the various elements that go to make up advice agencies would both constrain what it was possible for advice agencies to do, and would also produce values, actions and practices. As Hacking (1986, 223) notes, “counting is no mere report of developments. It elaborately, even philanthropically, creates new ways for people to be”. In considering the productive role of those fragments of the advice mosaic – the funding streams, volunteers, support systems, physical buildings, knowledges, technical procedures, and so on – and how they are intertwined with each other in the everyday practice of delivering advice, we began to think in terms of an advice assemblage.
\par
The concept of ‘assemblage’ emerged from the work of Deleuze and Guattari (1988), an assemblage being the dynamic functioning of heterogeneous elements as a seemingly stable whole. ‘Assemblage thinking’ (Li 2007) has been taken up by a range of scholars across the disciplines in recent years (see in particular Delanda 2002, 2006) who seek to understand how heterogenous elements come to have an appearance – and reality – of coherence and consistency (Deleuze 2006). We would emphasise five key elements to assemblage thinking that have informed and inspired our interpretation and analysis of our research data.
\par
First, the term assemblage can be used as both noun – the CAB as an assemblage – and verb – the acts or processes of assembling. As verb it focuses attention on the work or labour involved in constructing the assemblage: ‘the idea that the institutionalisation of specific projects involves work of assembling diverse elements into an apparently coherent form’ (Newman \& Clarke, 2009, 9). 
\par
Second, in paying attention to the components of assemblage we must also consider the ways in which these components interact: how the relationships between components, and between components and the whole, produce regulatory effects, shaping and reshaping the conditions, possibilities and problems of the assemblage; a focus on “not just how agency produces resultant forms, but on how the agency of both the assemblage and its parts can transform both the parts and the whole” (Anderson, et al., 2012,186).
\par
Third, time and place matter. An assemblage involves processes of transformation and movement, requiring us to understand the components not just for what they are now and how they relate to each other now; we must also pay attention to their historical relations (e.g. Randalls 2017). Even as assemblages are constructed or re-constructed, some elements continue to have historic associations that may be either productive for the new configuration – or may be disruptive or difficult to integrate. 
\par
Fourth, in working to create an assemblage, actors are making claims on/for a territory (Anderson and McFarlane 2011, 126). Here territory can refer to a geographical space or to a field of social practice where the assemblage claims ‘this is our business’. This ‘claiming’ becomes closely linked with assemblage as ethos or a particular orientation to the world. 
\par
Fifth, the assemblage (as opposed to its often-close relation ‘network’) carries with it the idea of movement rather than fixity, of fragility and instability as well as strength and stability (Clarke, et al., 2015, 51). An assemblage might be like a Lego construction, where one piece is adjoined to its neighbours by a specific fixing device; but equally, we could be looking at an assemblage where each new connection has to be thought about, invented and made stable – and might perhaps be destabilised by the adjoining of the next piece. 

\chapter{An Advice Service as an Assemblage: The Data}
Examining a local Citizens Advice office as an assemblage encourages us to look not only at the elements that go into making up the local office – the volunteers, spaces, managerial and administrative staff, and so on - but at the processes which bring these heterogeneous elements together in a particular time and place. We describe two key components in a voluntary advice service, namely the volunteers and funding streams, not simply in terms of their characteristics – who volunteers are, how much a funding contract is worth – but in terms of how they form part of the ongoing functioning of the advice assemblage.
\par
In this section we present the data gathered through the New Sites of Legal Consciousness programme. We start from the frontline volunteers who carry out much of advice delivery, addressing how managerial and audit structures shape this work, progressing to the funding streams that in turn both enable and shape the delivery of advice. We focus on these two areas for a specific reason: volunteers and money are often considered the two key ‘resources’ that keep the voluntary advice sector running. Yet the concept of ‘resource’ implies neutrality; something that can be inserted and removed from an organisation at will. We emphasise how these components occupy a specific place, shaped over time and in relation to other components, within the advice assemblage, and how changing dynamics in these two areas bring to light the changing role played by the advice sector.

\section{The Use of Volunteers}
Emphasising volunteer workers as a ‘component’ in the advice assemblage implies change in thinking: the volunteer workforce cannot be considered in isolation, but rather through its interactions and connections with other components as these have been shaped over time. Thus we discuss here the changing relationship between voluntary and paid labour, how the voluntary provision of advice is ‘held in place’ by systems of management and audit, and finally, emphasising the fragility of this component, its changing relationship to various aspects of the welfare state.
\par
Citizens Advice at the local level relies heavily on the body of volunteers who make up the majority of the front-line advisers. Indeed, the principle of voluntary (or peer-to-peer) provision of advice is extremely important to the identity of the service, and its claiming of territory with regard other ‘public’ services. With a few exceptions, volunteer advisers do not have formal legal training. Citizens Advice invests considerable time and personnel in making volunteers into experts who can deal with the wide range of problems that clients bring to the advice office, most of which have a legal element. Before taking on casework ‘solo’, volunteers go through a six-month period of extensive training followed by a period of observation by experienced advisers. Volunteer advisers are generally (though not always) carrying out ‘generalist’ work and are supported by salaried, specialist advisers who can take on complex casework and consult ‘generalists’ on specific issues, and by ‘duty managers’ and other managerial staff who oversee the general management of cases.
\par
As noted above, the voluntary delivery of advice was a key element of the identity of the Citizens Advice service as it was initially conceived. The distinctions between ‘voluntary’, specialist and managerial labour were the product of the professionalisation of the service in the 1970s and ‘80s (Citron, 1989), moves that were strongly resisted by those who strongly adhered to the classical model of voluntary, peer-to-peer support. During our research it was clear that funding cuts were shifting the dynamic interplay of these components. One CAB had employed a solicitor, Suzanne, funded by Legal Aid for employment work. Following cuts to legal aid they were forced to realign her role towards training and supporting volunteers to deliver employment advice. However, she identified difficulties with asking volunteers to fill the gap:
    \blockquote[Suzanne; specialist adviser, urban CAB]{I had a look at the audits, and they want a lot from these volunteers, and a lot of them I think just don’t have the capability, even in these one-off appointments, and it doesn’t necessarily look at what resources you’ve got to take things forward. In a sense it just assumes you’ve got resources to then do further work for them, and in our Bureau we don’t necessarily have that. There’s been cuts all around.}
Suzanne emphasises here how voluntary labour is impossible without the other components – the specialist and managerial labour and financial support that enable them. Yet it highlights also that, whilst volunteers create possibilities, they come with a range of complexities. Volunteers give their services voluntarily, not through any form of contract of employment. If there is a ‘contract’ it is a seemingly fragile one, based as it is upon the volunteers’ motivations for wanting to do the work. Managing a volunteer workforce through externally imposed targets is a risky business. Similarly, a change in the volunteer role, or in the demands placed upon them, or the way the CAB is run, may not fit with the ways in which volunteers understand and make sense of their work. Volunteers can face difficulties and anxieties regarding their role:
    \blockquote[Author field notes]{I worried about the possibility of having missed something or done something wrong. The greatest shock for me in the transition from the training has been the level of support provided – the extent to which this very much depends on who is Duty Manager at that time. I felt on this occasion that I was very much on my own. Another trainee had mentioned to me that they dreaded opening the notes book in the morning as it is here that the person checking over your case would leave their comments.}
What managers and advisers also emphasised was the importance of emotional support given the intense pressures of advising: As one trainer stated, “I think the pressure on advisers now is huge - huge.  ... I feel that they’re under a lot of pressures, a lot of work, there’s a lot of expectation.” (Romilly, Trainer in a semi-urban bureau).
\par
Indeed, the process of advice giving was described as deeply distressing for advisers, as my own field notes frequently noted:
    \blockquote[Author field notes]{The client has stayed with me, and indeed I can’t imagine going on advising. I have a week’s break now. Both the manager, who I like a lot, and the benefits specialist, who had in effect told me to be more heartless with the client, i.e. don’t give false expectations, were sympathetic afterwards, which helped a lot. The client broke down several times into uncontrolled tears and took an antidepressant half way through. The implication from the benefits specialist was that the client had been through all of this before with other advisers, that there was an element of manipulation and that it wasn’t fair on me. I don’t think this is entirely true, the client was in a genuinely impossible situation, one precipitated by factors largely outside of their control. I repeatedly realised that I didn’t have the appropriate responses.}
As a result, volunteers may choose to withdraw their ‘labour’, or simply just not follow directions. As such, performance management, and quality monitoring, of a volunteer workforce is a particularly difficult task (Paton 2003). In Citizens Advice, each bureau has its own management process where duty managers check and survey the work of advisers, with ‘behind the scenes’ or ‘upstairs’ support and management from admin and managerial staff. 
\par
Key to how the advice assemblage coheres, are the means through which the practice of volunteers is monitored and shaped by the auditing process. Indeed, assuring a quality service is key to Citizens Advice claiming and maintaining their territory. During our research Citizens Advice was in a period of change, introducing new centralised databases which recorded all client interactions and enabled more efficient and comprehensive case management. Crucially, these new databases enabled new forms of auditing. All those we spoke to were positive about new regimes and possibilities afforded by centralised case management systems, of maintaining control over cases and demonstrating the quality and value of the work, in the context of voluntary and as such non-specialist delivery of advice.
\par
Yet new auditing measures had particular effects upon how cases were carried out, as one manager explains:
    \blockquote[Stuart, manager in a rural bureau]{My own viewpoint is it’s business as usual, there should be nothing an audit should find that you don’t know already and I have done auditing work in the past myself and it’s…you want to see it, whatever you’re auditing for whatever reason, you want to see them as they are and not have someone kind of false presentation of how they think they should be and I know that I probably do a bit myself, everybody rushes round trying to get ready for an audit and make sure that everything’s tidy and neat and everything’s where it should be and all the ‘i’s are dotted and the ‘t’s are crossed and everything’s fine.  But you should be doing that on a daily basis, that should be normal life, we’ve got a standard to follow and we should be following that all the time.  So that’s my viewpoint on it - it doesn’t always work but that’s what you try to achieve.}
The voluntary provision of advice is perhaps the most important component of the Citizens Advice service. How this voluntary practice continues to function, as it does up and down the country in a variety of different spaces and different atmospheres, is only understandable in relation to the knowledge developed and maintained through training practices, the emotional and technical support of managerial staff, as well as the practices of self-monitoring established by processes of audit. Yet perhaps most importantly, what explains the endurance of this voluntary practice are the moral commitments of the volunteers. Key to understanding this commitment is the relationship of this component to an external factor, namely the welfare state. 
\par
\textit{Why} advisers continue to give their time and energy is tied to the experience of assisting people in difficult situations. Yet what is clear from our ethnographic work is that this attachment to advice clients is tempered by the feeling that advisers are doing work that should be carried out by welfare services, and secondly, that many solutions to life problems are no longer available. The attachment to the client, based on the possibility of improving their situation, is rapidly shifting in the context of the political changes described above.
\par
The fragility of this attachment, and as such of the voluntary labour component, poses significant problems for the service. While the ‘quality’ of the work is ensured by managerial processes and audit, the attachment of advisers to their clients and process of advising appears much more uncertain in the context of broader changes to the advice climate. As clients become more desperate and their problems more intractable, advisers increasingly voiced despair and hopelessness regarding the impact of their work, naming the significant areas in which they were not able to help and the feeling that they were filling in for services that were statutory duties of the state.
\par
Over time, the advice assemblage, itself developed in connection with the welfare state, has become a component of that welfare state; playing a key role in enabling individuals to access services and social protections. Yet the specific ways in which the central component within the advice assemblage – its volunteers – relate to the welfare state under conditions of austerity is rendering the assemblage distinctly fragile. The difficulty of maintaining the attachment of volunteer advisers to advice practice is placing the assemblage as a whole, or moreover the everyday ways in which the service is assembled, in distinct danger. 

\section{On Funding, Funders and Regulatory Effects}
Funding of public services is, of course, a key consideration in the United Kingdom in the context of widespread programme of funding cuts. Most important for our purposes are the swingeing cuts enacted by Central Government to Local Government funding, which has forced Local Authorities to re-consider their funding of advice agencies.
\par
However, while there has been considerable focus upon cuts to budgets, less well addressed is how the nature of funding affects the services delivered. In this section we explore how changes to the ways CAB are funded have fundamentally altered the nature of the advice assemblage. 
\par
Historically, Citizens Advice offices have relied almost totally on funding from Local Government. This situation is now dramatically changed with each local office drawing on funding from a wide range of sources, as this manager explains:
    \blockquote[Jill, manager, semi-urban CAB]{Our income this year is around £260,000.  Around £40,000 of that is from the district council and another £45,000 from the county council.  So £85,000 from local authorities, direct, it was in the form of a grant, we’re not working under any contracts, they are both service level agreements at the moment, although we are looking to be going to a commissioning process with one of them next financial year.  We also have a contract which originally was with the Primary Care Trust .. at the moment that funding sits within the public health body of the county council .. to deliver welfare benefits advice .. around £45,000 a year.  That’s only confirmed until March after which we are advised .. we will have to commission and we already know that there is some competition out there for that so that’s a little unsure – and that’s for specialist welfare benefits.  We have a contract with the Money Advice Service .. to deliver debt casework and that is probably at the moment our most secure funding because we have .. we’ve got three years more .. That’s it as far as our major funding.  The rest of it comes from small pots of services. We deliver two outreach services, one for Sure Start Children Services – so again, through the local authority but specifically sits with children’s services.  And we deliver general advice at outreach in four of the children’s centres across the district.  That’s just £15,000 to do that.  And I’ve recently secured some money from the RAF Benevolent Fund, to do outreach at three of the RAF bases in the district.  The rest of the money comes through donations from parish councils, town council, our own fundraising efforts and then mainly applying for small grants.}
Jill’s experience, managing an advice office located in a town also serving the surrounding rural area, is typical of those we interviewed, most particularly in the English CAB. We can see that a CAB’s budget can look like a mosaic, made up from a diverse range of funding sources: statutory, charitable and business. For Rhea, who manages a large city-based office with an annual income of £650K, of which only 29\% came from the local authority, such diversity was a deliberate strategic choice:
    \blockquote{It’s a diverse funding base and we have a principle of trying to retain as diverse a funding base as we can. Survival seems to depend on that these days. We have what we glibly call core funding from [Local City Council] [Rhea, manager in an urban bureau].}
So whilst local authority funding is now a much lesser component than it had been in the past, it is nevertheless seen as vital ‘core funding’. These local authority grants were generally not directed towards delivering specific forms of advice or to specific categories of people (except certain restrictions about the extent to which out-of-area residents could be helped). Rather this LA funding supported general management and building costs, and enabled the provision of a generalist advice service which is so key to the Citizens Advice principle of providing a holistic approach to advice need. It also enabled local advice offices to meet advice needs that it was difficult to obtain specific funding for: at least one CAB in our study provided specialist advice in employment through the local authority core funding. 
\par
It was not just the quantity but the form that funding took that was key. Jill, Rhea and others emphasised the importance of the local authority funding being a grant and not a contract, a distinction that is recognised as key in the voluntary sector literature:
\par
    \blockquote[Osborne and Waterston, 1994, 294]{[A] grant could ideally be defined as an arrangement which is a general contribution to an organisation, and is not intended to support an identified output of an organisation, .. By contrast, a contract is an arrangement which is made for the production/delivery of a specified output, with the magnitude of the payment determined by an agreed price to the client of this output.}
Whilst grant funding has been the traditional form in which local government provided support to the voluntary sector, Jill’s quote demonstrates how councils now look for mechanisms that enable specification of targets against which performance can be evaluated. The contract is the mechanism through which the logic of markets and the business model can be streamed into the voluntary sector. Contracts bring with them a range of regulatory mechanisms that reshape the work of advice. In particular, local authorities were shifting from ‘open’ grant funding to contracts that tightly specify the service to be provided, whilst bureaux, seeing other funding restricted, had become increasingly reliant, as Jill notes, upon the contract with the Money Advice Service to provide debt advice. In this light, we approach each contract as a component in the advice assemblage; shaping it, maintaining it, and potentially destabilising it. 

\section{Legal Aid Funding}
Civil legal aid, introduced in England and Wales in the mid-twentieth century, was once considered one of the “great pillars of the post-war welfare state” (Lord Beeching quoted in Moore and Newbury (2017): 17). Legal aid cuts, enacted through the Legal Aid, Sentencing and Punishment of Offenders Act 2012, had a dramatic impact on those who had become reliant on legal aid funding to provide services (particularly law centres, see Mayo, et al., 2015). A few local Citizens Advice offices in our study had funded some work through legal aid. As these funding streams had ended prior to the time of our interviews, our participants’ reflections related to the impact of these cuts and their implications for the advice work at the local level. 
\par
Legal aid funding was tied into a contract model; it required CAB to meet particular client needs for advice. One urban CAB had received legal aid funding for debt, welfare benefits and employment advice which enabled advisers to represent clients at benefits appeals tribunals. Loosing legal aid funding meant clients having to do more themselves:
    \blockquote[Miriam: Specialist Adviser in an urban CAB] {[We] used to have specialist benefits work where we could prepare and represent at appeals – that’s gone because that was funded by legal aid.}
However, whilst legal aid enabled the provision of particular types of advice, the contract imposed limitations on the possible practices of advice. Local authority funding had allowed advisers to make decisions about client’s support needs that a target-driven contract pulled them away from. Monitoring and performance targets placed severe burdens on advice offices to the extent that exiting from legal aid contracts was taken as a positive decision:
    \blockquote[Grace, Focus Group, urban CAB]{When I was a money adviser, we were funded by the Local Authority. There wasn’t any Legal Service funding. We were just given the grant to have money advice and there were no targets, there were no times set, there were no … we were just completely client driven.}
As Jill notes, having ended Legal Aid funding on their own terms, the bureau did not have to deal with the consequences of their sudden removal:
    \blockquote[Jill, Manager, semi-urban CAB]{We got out of legal aid contracts before they changed because we had a situation where it became apparent that the contracts financially were not giving us anything.  We were subsidising them with volunteers and paid staff.  And we were fortunate enough to secure alternative funding to deliver the benefits advice that we were delivering under the legal aid contract … It was the best thing we ever did because we weren’t involved in … losing the majority of income that a lot of the CABs have just gone through.}
Legal aid cuts were only one aspect of the increasing pressure on CAB resources in the first two decades of the twenty first century. Pressure on local authority budgets arising from the politics and policies of austerity of the Tory/Liberal Democrat Coalition of 2010, was passed on in cuts to voluntary sector funding and a shift towards more tightly specified contracting, leading local advice offices to seek funding from a multiplicity of resources described in the opening to this section.

\section{The Funding Mosaic: Entrepreneurialism and Independence} 
From being principally funded by local government, most Citizens Advice office budgets are now a mosaic of statutory, charitable and business funding. As those organisations who had become heavily reliant on legal aid had experienced, over-reliance on one source could have disastrous implications, diversity had become a virtue. In addition to the major funding from Local Authorities and the Money Advice Service, some of the funding sources named in interviews were: Macmillan, to deliver advice to cancer patients and their families; Royal British Legion, for advice to ex-servicemen and their families; local water boards, to help with the large numbers of clients with water debts; electricity companies, for debt advice but also to assist with energy use; and the National Lottery, for a range of specific tasks including advice service transition funding which was meant to help the advice sector cover the gaps left by legal aid funding.
\par
As we saw above, funding acts as an enabler, providing resources to employ staff or train volunteers to deliver advice services. However, funding also regulates the organisation: funding sources that are targeted towards specific actions or goals or population groups can enable advice to be delivered in new ways or to a previously neglected client group, but at the same time require the organisation to direct resources to these tasks. In the process, the holistic approach to advice is challenged by the need to gain funding which only comes tied to specifically identified needs. 
\par
Many funding opportunities are ‘project’ based rather than ‘general’; that is, they concern the delivery of a specific advice service to a pre-defined group. There was a certain amount of cynicism amongst interviewees about project-based funding: what you can get funding for is ‘dictated with what’s going on at a national level’ [Eddie: Focus group in an urban CAB]; and, whilst you can get funding for a specific project it then becomes ‘yesterday’s issue’ [Marilyn: Generalist adviser in an urban CAB]. Identifying and successfully bidding for project-focused funding required particular skills and expertise in managers:
    \blockquote[Marilyn: Generalist adviser in an urban CAB]{[O]ur manager … he’s very, very good at getting funding for specific areas and advice which is why we have a Housing Advisor, why we have an Armed Forces serving veterans advisor, why we have a Prison’s Advisor, why we have an employment advisor, why we have a patient...and NHS complaints advisor. All of those people are on specific pots of funding for specific projects.}
The changing nature of the funding component in the advice assemblage is not only reshaping what CAB are able to do, it is also altering the skills required of those who staff the local offices. Being a CAB manager now requires entrepreneurial skills, the ability to spot the opportunities and expertise in managing relations and networking; directors/CEOs are involved in a constant round of grant bids. This new set of skills gives more managerial attention to the external world of the organisation: scanning, liaising, building partnerships and becoming competitive with other potential providers. It also means becoming skilled at bid writing and the management of more-or-less-tightly specified project grants.
\par
These shifts in funding regimes to more target-driven approaches had implications for the perceived independence of the advice office. As one of the four ‘pillars’%
    \pagenote{The others being that advice is free, confidential and \url{https://www.citizensadvice.org.uk/about-us} [last visited 26/11/2015].}
of the Citizens Advice service, its ‘independence’ of its advice was frequently identified by advisers and managers as a central element in their attachment to the service; the principle of independence is critical to claiming territory. The following exchange, which took place in a focus group of advisers in an urban office, relays some of the complexities of remaining ‘independent’:
    \blockquote{\textnormal{\bfseries Amelia}: Yes but we don’t act for… Like for example you get [X building society] referral, [X building society] would pay us to see that person, but we will often act against [the same building society] for example for possession proceedings. [\textellipsis]
    \\\textnormal{\bfseries Joe}: We might be paid by the Council to give advice to people that have got Council Tax issues, but they’re not dependent - the funding is not dependent on us making sure that we don’t write off those debts through insolvency, for example, so it’s not funding for an outcome that is specific to protect that funding. 
    \\\textnormal{\bfseries Amelia}: But on the other hand, we would tell them to pay their Council Tax. 
    \\\textnormal{\bfseries Joe}: Oh yeah, yeah, because it’s a priority [debt], but not because the Council are saying we don’t want problems with Council Tax.}
However, as another adviser expressed, targeted funding can make advisers feel in a ‘very difficult position’:
    \blockquote{\textnormal{\bfseries Fred}: sometimes I’m being put in a position of some sort of proxy rent collector for the Local Authority and it’s very, very difficult sometimes not to get into that mind-set, particularly if the follow up from the initial appointment is the Housing Officer badgering me by email or telephone: ‘oh, they’ve not paid their rent this week’. And I’m thinking, that is not my job, that’s your job. I’ve advised them about what they can do. I’m implementing the strategy that we’ve agreed. If the tenant is still not choosing to pay their rent, then that is their choice and they know the consequences.} 
‘Independence’ frequently crops up as a point of particular anxiety in the voluntary sector, most often expressed in the Third sector literature as a need to be independent from the state (e.g. Lewis 2005), operating in the ‘specter of state capture’ (Bedford 2015, 464; also see Dunn 2008; Musick \& Wilson 2008; Rochester, Paine and Howlett 2010) or as part of the ‘shadow state’ (Wolch 1989; Jones 2010). Most recently, the UK government has overtly used funding contracts to restrict the ability of organisations to challenge and dissent, introducing a requirement that central government grant agreements must include a clause which prevents these funds from being used to lobby or attempt to influence parliament, local government or political parties.%
    \pagenote{\url{www.gov.uk/government/news/government-announces-new-clause-to-be-inserted-into-grant-agreements} [last visited 10.04.16].}
\par
However, regulatory effects are rarely so transparent. As the exchanges above demonstrate, the (re)aligning of interests of funder and fundee, creating conditions of dependency (see e.g. Clarke, et al., 2010, chap 6), are experienced and inhabited in complex and shifting ways, just as earlier we showed how the volunteer workforce also has ways of shaping what is, and is not, possible.
\par
There is a tendency to consider funding in neutral terms; as a flow of capital that determines whether a service can be delivered or not. Here we have presented ‘funding’ as a more nebulous category of separate funding streams, each of which can be seen as a ‘component’ of the advice assemblage with distinct effects upon the ongoing practice of advice. The funding contracts described by our participants were characterised by precariousness, requiring continual reapplication. Indeed, even local authority funding is no longer a given; some bureaux have been forced to close as a result of losing their local authority grant. We would draw attention to how this shift towards a precarious funding environment has effects upon other areas of the service, in particular inasmuch as they draw the energy and focus of managerial staff away from volunteer support. 
\par
Yet perhaps the most important implication of this shift towards funding ‘mosaics’ is the recognition that individual funders have their own purposes for funding advice, whether it be facilitating rent or council tax payments, or assisting a particular group, and the ways in which this has been experienced as compromising the independence of the service. That is, while enabling advice that otherwise would not be given, certain funding contracts have been experienced to affect other components and attachments that keep the advice assemblage together.

\chapter{Fragile Futures}
The value of the analytical tools that assemblage thinking offered became particularly clear when we began to worry about the real possibility of advice agencies falling apart under the threat of cuts in key funding. For, as Newman \& Clarke, have observed, assemblage
    \blockquote[Newman \& Clarke, 2009, p.~9]{draws attention to the work of construction (and the difficulties of making ill-suited elements fit together as though they are coherent) … makes visible the (variable) fragility of assemblages – that which can be assembled can more or less easily come apart, or be dismantled.}
In our research with Citizens Advice we became increasingly impressed by two factors: first, the incredible commitment on the part of all those who worked for the service, despite the multiple changes, difficulties, frustrations and anger they faced on a daily basis; but second, the fragility of this service that has been assembled over the decades, an assemblage which appears perhaps to be an iconic British institution, but which could, more or less easily, come apart.
\par
The threats are multiple, and have become ever-more acute since we finished our interviews and focus groups. The assemblage that is the local Citizens advice office is interconnected with many other assemblages, particularly assemblages of local and national government, all of which are pulling in multiple directions, forcing “reordering boundaries between practices” (Mellaard and van Meijl 2016, 12). 
\par
Under the pressure to do more with less, one threat arises from the increasing emphasis on phone-based and digital advice, leading to a move away from the face-to-face, one-to-one relationship between adviser and client. The threat here is to the holistic ethos of the advice service, which was seen by almost all we interviewed as being central to the Citizens Advice approach. An understanding that the role of the adviser, in the first instance, was to listen to the person in front of them, a listening which would often mean that the problem or concern that was presented as the problem by the client most frequently unravelled as one of many; that the threat of eviction arose from a history of debt which had arisen from a dismissal from work that perhaps could and should be challenged. Government officers at all levels place an increasing faith in the digital as the solution – in relation to advice-giving, most advisers would advise caution. 
\par
A second threat is to the social policy campaigning function of Citizens Advice from a government that appears to want to close down all debate and dissent; a ‘Big Society’ that means the only role for the charity sector is to deliver services government has decided it can no longer afford. The clause to stop third sector organisations using government grant funding for ‘lobbying’ (detailed earlier) has been condemned by charity organisations and some MPs, as threatening the sort of social policy work carried out by Citizens Advice, “to bring real-world experience of service users and evidence-based expertise into the public policy debate”.%
    \pagenote{\url{www.thirdsector.co.uk/commons-motion-urges-government-reconsider-anti-lobbying-clause/policy-and-politics/article/1385085} [last visited 10.04.16].}
As we emphasised, the social policy functions of the service are key to how advisers understand and make sense of their work; the truth telling they must engage in with clients must be balanced by a service that is able to speak the truth to power.
\par
The third threat also arises from the ideology of the present Conservative government (and its predecessor Coalition), in its unrelenting dismantling of local government.%
    \pagenote{One CEO told us that the local authority was drawing up two lists of the services: those for which they had a statutory responsibility, and those where there was no statutory responsibility. They were considering only funding those on the first.}
For whilst most of those we interviewed identified that local government funding was becoming less important in actual volume terms, they all identified the crucial importance of what they called ‘core’ local authority funding. This in effect supported the ‘heart’ of the organisation, paying for office space, administration, directors and other ‘backroom’ support; without this funding it would not be possible to bid for the project-driven funding from charitable, statutory and private sources because there would be no-one funded to do this entrepreneurial work, and there would be no premises from where the advisers – voluntary and paid – could deliver the services. Yet perhaps more importantly, it is this funding, typically the only funding not tied to a particular problem area or client group, that guarantees that local CAB can provide generalist advice. 
\par
By using our ethnographic work to present two components of the advice assemblage, namely volunteer workers and funding streams, we have focused less in this paper upon these direct threats than the ways in which these combined changes are fundamentally shifting central components of the advice assemblage. We have noted how these changes have inscribed a fragility and uncertainty in this assemblage that is not always visible; indeed increasing demand for the service gives the impression of it being as stable as ever. We noted first how voluntary provision of advice is dependent upon the moral and normative commitments of advisers, in particular inasmuch as these commitments relate to the provisions of the welfare state, that is increasingly difficult to sustain in a climate of ‘austerity’. We noted also how The diversification of funding strands has further fundamentally altered the composition of the advice assemblage, rendering the entire edifice more unstable and uncertain. In addition, the general shift towards from a ‘grant’ to a ‘contract’ funding model has been experienced to affect the ‘independence’ of advice.
\par
Amongst a great many who operate outside the voluntary sector there is a strange faith that the voluntary sector can survive no matter what – that voluntary sector organisations exist because of the voluntary effort. In relation to Citizens Advice, whilst all our research points to the central importance of the voluntary effort, it also identifies the inextricable interrelatedness of the volunteers and public funding. The assemblage that is the advice service is a fragile structure – take one key element away and the whole device could crumble, collapse or melt away.