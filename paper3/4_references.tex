\label{paper3:references}
\begin{thebibliography}{99}
%%%%%%%%%%%%%%%%%%%%%%%%%%%%%%%%%%%%%%%%%%%%%%%%%%%%%%%%
%%%%%%%%%%%%%%% START 9_references.tex %%%%%%%%%%%%%%%%%
%%%%%%%%%%%%%%%%%%%%%%%%%%%%%%%%%%%%%%%%%%%%%%%%%%%%%%%%

\item Abbott, A. (1988). \textit{The System of Professions. An Essay on the Division of Expert Labor}. Chicago: The University Chicago Press. 
\item Aarestrup, A. K., Due, T., Drud, \& Kamper-Jørgensen, F. (2007). \textit{De kommunale sundhedspolitikker i Danmark - en kortlægning}. Odense: Statens institut for folkesundhed og Syddansk Universitet. 
\item Barter, C. and E. Renold. 2000. ‘\textit{I Wanna Tell You a Story’: Exploring the Application of Vignettes in Qualitative Research with Children and Young People}. \textit{International Journal of Social Research Methodology }3(4):307–23.'
\item Charmaz, K. (2006). \textit{Constructing grounded theory. A practical guide through qualitative analysis} (2nd ed.). London: SAGE Publications. 
\item Etzioni, A. (1969). \textit{The semi-professions and their organization: Teachers, nurses, social workers}. Free Press.
\item Evetts, J. (1999). Professionalisation and professionalism: Issues for interprofessional care. \textit{Journal of interprofessional care}, 13(2), 119-128. doi:10.3109/13561829909025544.
\item Evetts, J. (2011). A new professionalism? challenges and opportunities. \textit{Current Sociology}, 59(4), 406-422. \url{doi:10.1177/0011392111402585}.
\item Foucault, M. (2000). \textit{Klinikkens fødsel} [\textit{Naissance de la clinique}]. København: Hans Reitzel.
\item Freidson, E.,1923-. (2001). \textit{Professionalism : The third logic}. Cambridge: Polity.
\item Freidson, E., \& Lorber, J. (Eds.). (2008). \textit{Medical professionals and the organization of knowledge}. London: Aldine Transaction. 
\item Goffman, E. (1963). \textit{Behavior in public places. Notes on the social organization of gatherings}. New York: The Free Press. 
\item Goffman, E. (1990).\textit{The presentation of self in everyday life}. Harmondsworth: Penguin.
\item Harrits, G. S., Larsen, L. T. (2016). “Professional claims to authority: a comparative study of Danish doctors and teachers (1950-2010)”, \textit{Journal of Professions and Organization}, 0:1-16. \url{doi: 10.1093/jpo/jov011}.
\item Hill, Michael (2005). \textit{The Public Policy Process} (4th edition), Essex: Pearson Education Limited.
\item Miles, M. B., Huberman, A. M., \& Saldaña, J. (2014). \textit{Qualitative data analysis: A methods sourcebook} (3rd ed.). Thousand Oaks, CA: Sage. 
\item Møller, M. Ø. \& Elmholdt K. T. (2018). “Den offentlige organiserings materialitet: Et studie af fysisk organisering og interaktion i lokale sundhedshuse”, \textit{Politica}, 50(3): 384-401. 
\item Navarro, V. (1976). \textit{Medicine under capitalism}. New York: Prodist. 
\item Noordegraaf, M. (2015). Noordegraaf, M. (2015). Hybrid professionalism and beyond:(New) Forms of public professionalism in changing organizational and societal contexts. \textit{Journal of professions and organization}, 2(2), 187-206.
\item Parsons, T. (1975). The sick role and the role of the physician reconsidered. \textit{Health and Society}, 53(3), 257-278.
\item Pedersen, B. M. \& Rank Petersen, S. (Eds.). (2014). \textit{Det kommunale sundhedsvæsen} (1. udgave ed.). København: Hans Reitzels Forlag.
\item Sehested, K. (2002).“How New Public Management Reforms Challenge The Roles Of Professionals”, \textit{International Journal of Public Administration}, 25(12): 1513-1537. \url{doi:10.1081/PAD-120014259}
\item Schott, C., Van Kleef, D., \& Noordegraaf, M. (2015). Confused professionals?: Capacities to cope with pressures on professional work. \textit{Public Management Review}, 1-28. \url{https:/doi.org/10.1080/14719037.2015.1016094}.
\item Schreier, M. (2012). \textit{Qualitative Content Analysis in Practice}. London: SAGE.
\item Sundhedsministeriet (2010). \textit{Sundhedsloven}. LBK nr. 913 af 13/07/2010. København: Sundhedsministeriet.
\item Turner, B., S. (2007). \textit{Medical power and social knowledge} (Second Edition ed.). London: SAGE Publications. 

%##################################################################################################################################################################################################################################
\end{thebibliography}