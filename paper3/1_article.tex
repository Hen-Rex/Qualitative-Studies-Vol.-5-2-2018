%%%%%%%%%%%%%%%%%%%%%%%%%%%%%%%%%%%%%%%%%%%%%%%%%%%%%%%%
%%%%%%%%%%%%%%%%% START 1_ARTICLE.tex %%%%%%%%%%%%%%%%%%
%%%%%%%%%%%%%%%%%%%%%%%%%%%%%%%%%%%%%%%%%%%%%%%%%%%%%%%%

%\chapter{Introduction}
\lettrine[lines=2]{\bfseries\color{black}I}{n recent years}, there has been a policy trend for western welfare states to decentralize health responsibilities to lower levels of government. In Denmark this shift occurred as a delegation of health promotion and rehabilitation activities from regionally governed hospitals to local municipalities (Pedersen \& Rank Pedersen 2014: 276; Sundhedsministeriet 2010). Local governments were now to focus on health activities targeting health promotion and rehabilitation, paving the way for not only new relations between health professionals and citizens, but also for a more relaxed staffing policy regarding who should deliver these health services to citizens. Local health activities were no longer bound to the hospital service or to the medical field, but were defined much broader as health promotion and rehabilitation (Aarestrup et. al., 2007: 1). Because of this policy shift, many municipalities chose to organize health promotion and rehabilitation in health houses, employing health professionals from professional colleges instead of medical doctors from research-based universities. However, it remained politically and administratively unclear what exactly was the purpose and the content of local health responsibilities in health houses and because of this, it has not been possible to derive any clear picture of what characterizes these new encounters between health professionals and citizens without theoretically and methodologically zooming in on how concrete interactions play out in these new institutional settings. The policy shift represents an opportunity to study such interactions including a way to understand more carefully processes of professionalization. 
\par
Therefore, the question addressed in this article is whether these new local health houses comprise an alternative to the medical view on health or---even in the absence of the hospital---script the professionals to identify themselves as agents from the medical field? More specifically, how do these new material settings matter to health professionals working in the health houses? The article shows an example of how material settings script processes of identification and counter identification among health professionals and citizens, something that has not been largely explored so far.
\par
The article is structured as follows: First, a theoretical section that introduces the structural concept of professionalization and hybrid professionalism, the concept of counter identification as a social mechanism, and in accordance with symbolic interactionism, the idea of physical frames as inseparable from social processes of interaction. Next, a methodological section describes the research design, the methodology of the study and the qualitative techniques used followed by an analysis of observations and interviews illuminating how they see themselves in two distinct physical health house settings. Finally, a discussion of findings and theoretical implications is made. The article seeks to contribute on how to combine structural concepts of professionalization with micro relational theory. The empirical analysis may also have a methodological contribution on how to zoom in on few cases from a larger sample of evidence by following strict steps of data collections, case selections and data classification. These steps are described in a comprehensive appendix containing essential characteristics of the selected health house observations and interviews with health professionals and citizens in the larger empirical sample. 

\chapter{Theoretical Framework: Professionalization in Site-Specific Contexts}
In Denmark, as in many other western countries, the development of professions has been emphasized as important factors driving social change in society (Turner, 2007:189). Eventually, many professions, including medicine, which is important in this context, moved from being a merely practical to a scientific and a clinical discipline (Navarro, 1976). During the 20th century, professions became challenged by a more well-educated population much more conscious of their role as taxpayers as well as of their rights as citizens. This development created new forms of relationships between professionals, citizens and society, where professionals’ authority as experts was re-defined (but not necessarily declined, see e.g. Harrits \& Larsen 2016:9) along with the invention of new forms of service activities (Turner 2007).
\par
Theory of professionalization focuses on such historical developments in order to identify which social and political factors influence the direction and changing content of professions. A key argument is that the larger social and political context is inseparable from the development of professions, and that especially governments (state-level) and the bureaucracy of public service (organizational level) are seen as driving mechanisms for professionalization processes and the development of professions (see e.g. Seehested 2006). Hence, professions and societies are seen as intertwined, the increase in complex political and administrative systems influencing not only the demand-side of professional service, but also the working conditions of professionals embedded in politico-administrative bureaucracies. Today, many professions, health professionals included, are also tightly coupled with the public sector. Within the literature of professionalization the idea is that this intertwined relationship between the public sector and the professions has added an extra layer of organizational tasks to the assignments of many professionals (Seehested 2006). Professionals need to be able to treat, diagnose and reflect based on professional knowledge, but also to decide and structure interventions in accordance with political and administrative goals (Abbott 1988; Hill 2005: 274). To grasp this development, newer studies distinguish between ‘organizational professionalism’ and ‘occupational professionalism’ to identify effects of external frameworks such as regulation of professionalism in bureaucratic frontline work (Evetts 1999; Schott, van Kleef et al. 2015). They describe a development where professionals at first experienced bureaucratic working conditions as a pressure, then later seeing them as enhancing their room of professional manoeuvre to act, reflect and treat. This idea of organizational and professional capacity building emphasize that professionals’ skills must include both organizational and occupational dimensions or what has been coined ‘hybrid professionalism’ (Noordegraaf 2015; Evetts 2015). In line with this, and as originally pointed out by Freidson, these professionals hold the key to better accountable control of public policy service delivery (Freidson 2001). In this sense, professionalism represents a ‘third logic’ capable of overcoming both consumerism and bureaucracy in legitimate and effective ways. When professionals achieve organizational competences, organizations can standardize working procedures and at the same time motivate professionals to maintain frontline responsiveness. Hence, to some extent the development of ‘hybrid professionalism’ is seen as something that has the potential to include the best from the world of managerialism and the world of professionalism as a co-product of both occupational and organizational principles and values (Evetts 2011). However, this stream of research is mainly macro- and structural oriented and does not pay particular attention to the impact of the local institutional and physical setting of frontline work, which is something that is proposed elsewhere as significant to professionals’ identification with their profession and to the social process of professionalization (see e.g. Goffman 1990). Here, it is emphasized how professionals are supported, not only by the regulative framework and the organizational structure, but also by the physical setting around them. In a different study of Danish health houses, the architecture and the décor of health house locations were identified as important markers for the content of social interactions between professionals and citizens (Møller \& Elmholdt 2018). \par
In line with this, Michel Foucault in his study of the importance of the medical clinic in the transformation of medicine into a scientific discipline, and Freidson \& Lorber in their studies of medical professionals emphasize how the emergence of new professionalisms is associated with social setting and spatial surroundings (Foucault 2000; Freidson \& Lorber 2008). They argue that it is possible and even preferable to study macro structures in microcosm, or in specific locations and situations in order to point out how and why the larger social context inform (inter)action and reflection at the individual level. So, the hypothesis that spatial surroundings facilitate processes of identification is far from new, but a classic micro sociological hypothesis, first formulated by Erving Goffman (1963; 1990). He sees social interaction as ‘plays’ performed on scenes scripted by physical and social values. On this ‘scene’ people meet and interact drawing on (the) physically and socially available scripts. He sees the physical setting of any form of social interaction as a ‘scene’ that mirrors status roles and positions to be filled out by the ‘players’ performing on the scene. Interactions take place on the ‘front stage’, the ‘scenic aspects of the front’ (the setting consists of physical décor such as furniture, decorations and accessories) constituting the room of performance (Goffman, 1990:32). These signs and meanings encourage certain ways of performing while suppressing others. A scene varies in its degree of rituals, i.e. to what extent actions are performed without the need for subtle reflection or initiative. For example, there is a difference in the degree of ritualization between a surgery room in a hospital and an office waiting area. In a waiting area, people can move around more freely, yet reflective about where and how to perform (sit and wait), whereas performance in a surgery room is more or less fixed in advance by performative standards. The basic argument is that general social laws structure social encounters and that the physical and social settings have an effect on how people perform and behave towards each other (Goffman, 1990). The way the material scripts inform actions is hence an important dimension for understanding individuals’ reason for why they perform and act as they do in (the) encounters with others. 

\section{Identification and Counter-Identification}
In addition to understanding the impact of ‘physical frames’ on how health professionals think of themselves as professionals and of citizens as users of the health house, both Goffman and Freidson stress identification as being an important social mechanism in identity formation both on an individual and on a group level. However, as pointed out by Freidson and Lorber, identification with others does not necessarily need to be positive, as also negative perceptions of others can have the same powerful impact on how individuals and groups perceive of themselves and others through marking their identity and values in contrast to others. They use the concept ‘counter identification’ which refers to this reverse form of identification as an aversive identification with significant others. The concept denotes a social mechanism of claiming autonomy through distance and rejection of other people’s values and actions (Freidson \& Lorber, 2008). Hence, counter-identification denotes the social process whereby the rejection of another person’s perspective works as a prerequisite for self-identification, as, for instance, when a person dissociates him- or herself from other persons considered as pejorative in order to signal how he or she is different from them. Following this line of reasoning, the following analysis looks for whether health professionals use counter-identification to position themselves as different from medical doctors as a way to mark a difference. The current theoretical framework pays attention to the physical setting as it is expected to script how citizens interact as even objects tell you what is expected from you. A desk asks you to wait for your turn and a café invites you to have informal conversations and maybe to eat and drink something. The physical setting encourages certain professional routines and suppresses others such as exercise in the gym room or surgery at the hospital. Also, professionals’ backgrounds are site-specific: many health professionals are trained in hospital surroundings, but may end up working in spatial contexts outside the hospital, e.g. in a local health house that does not support standard medical routines. The same argument goes for citizens: Their expectations to professionals’ agency are routed in site-specific experiences that might structure the way they acknowledge or reject these new health professionals they encounter in the health houses. Therefore, citizens perspective will also be significant to include here as a window into understanding what values they associate or dissociate with the local health house. In this sense, citizens’ ‘judgment’ of the health house’s value may influence the way health professionals’ (counter)identify themselves with doctors or other professional roles. Following the argument that spatial setting signals behavior and identification it is expected that these new local health care houses influence how local health care professionals and citizens perform in the absence of doctors and in the absence of the spatial surrounding of a hospital. 

\chapter{Mixed Methodological Research Design: Case Selection and Vignette Construction}
In order to examine how health professionals see themselves in the health house it is important to know how and where to zoom in on these individual interactions. In order to gain this knowledge, a mixed method approach was used. First step was a telephone survey of all 98 Danish municipalities asking very descriptive questions on how the new local health responsibility was handled, followed by a classification of these answers in order to advance an overall picture of the variety of types of municipality implementation of local health care responsibility. The survey results showed that many (but not all) municipalities designed health houses, but in different ways. Some municipalities copied the medical clinic’s physical setting and staffed health houses primarily with nurses, whereas other municipalities created a more informal health house staffed with a larger variety of professionals. This survey knowledge was hence used to construct two small descriptions (vignettes) of health houses representing this main difference (see Barter and Renold 2000). One vignette described a health house accessed (straight/directly) from the street without a referral, whereas the other vignette described a more medically-clinically based health house, where you need a referral from a GP or a social worker to get access (see vignettes in appendix 1). Hence, vignettes were used to construct plausible organizational differences between health houses mirroring frames from an assembly house and a medical clinic respectively. In order to further zoom in on these health houses a second step of case selection followed. Based on a criterion of relevance (health houses must deliver local health services) and geographic variation 19 out of a total sample of 48 health houses were selected for observations and from them 36 health professionals and 22 citizens were interviewed (see appendix 6). Interviewees were qualified for selection if they provided or received health services from one of the houses (see convenience sampling strategy in Weiss 1994: 24-26). The vignettes were then integrated into the interview guide and used as a way to probe interviewees to reflect upon these different settings. In practice the two vignettes were written on small pieces of paper and given to the interviewee halfway through the interview followed by questions on what they thought about them. Their answers were hence coded and classified according to their health house preference and the associations they made to professionalism and health respectively. 
\par
In addition to these interviews (see interview guides in appendix 3), on-site observations were made in accordance with specific observation points, such as the physical order and structure of the place, persons present at the site and interactions at the site (see observation guide in appendix 4). The on-site observations were prepared as field notes. The interview data were transcribed, segmented and open coded to learn about the dimensions in the material (Schreier, 2012; Charmaz, 2006). This initial data analysis was followed by a focused coding (Miles, Huberman, \& Saldaña, 2014). Based on these classifications and coding of the interviews, analysis showed that the health professionals came from a broad variety of professional occupations (see appendix 2), and that they all had a professional college degree (BA level). Even though no GPs were working in the selected health houses some (nurses) were familiar with logics of ‘traditional medicine’, because they were trained in hospitals. In order to be able to further zoom in on social interactions in particular settings, a detailed memo was written, synthesizing the main values and attitudes put forward by each respondent (see appendix 5). Based on these memos, a third and final case selection of only two health houses was done. The purpose of zooming in on only two health houses was to explore the contextual details of how encounters occur in distinct types of health houses and how the physical frames inform how health professionals see themselves in this newly formed setting (see selected cases in appendix 6). Observations and interviewees from one of each of the main types classified from the same municipality were selected to ensure a comparable local regulative setting, however with different spatial surroundings. In the following, these two health houses (M1 and M2) located in the same region and municipality (M) will be analyzed. The purpose is to illustrate interactions in the spatial surroundings in a medical-clinical-based health house and a community-based health house respectively as well as to present how health professionals and citizens think about them.

\chapter{Analytical Section: Observations and Interviews in Two Distinct Types of Health Houses}
\section{The Medical--Clinical Health House: “Press If You Have An Appointment”}
The following description is based on on-site observation of M1. This health house is situated in a new residential area close to a major cycle path connecting the suburbs with the inner city. It is located on the ground floor of a newly built apartment block surrounded by other apartment buildings. There are parking spaces, bicycle sheds and a bus stop in front of the building. A lawn lies between the building and the cycle path. At the starting time of the observation (09:30 in the morning), there were no people in the area. There is a big sign displaying the name of the health house in front of the entrance. You have to go through two glass doors to enter. On the inner door is a sign saying that you are to announce your arrival on the flat screen placed right opposite the entrance. There is a small waiting area, called the café, just inside the entrance. Here, there are three tables with chairs and a coffee maker. On the left side there is a hallway leading to training rooms on one side and meeting rooms on the other. On the right hand side, there is also a hallway leading to more meeting rooms and an office. A set of red footprints marked on the floor guides you in the direction of the large flat screen. On the screen, it says, “Press if you have an appointment”, “Press if you have no appointment” and “Cannot remember who you have an appointment with”. When you press the button “Press if you have an appointment”, pictures of all the employees appear. The walls are white, and the floor is gray linoleum. There is a large gray mat in front of the entrance. There is a wall with a lot of brochures and pamphlets on various health services. In the café, there are a few brochures and a local newspaper. There is a painting on one of the walls.
\par
The overall impression of the surroundings of this health house is that it is a very quiet place without many people. This may be due to the time of observation, though one of the interviewees also confirmed that they have a problem of low demand. She says that people do not know that the house exists, and that people out in the “new suburbs” do not use it, because it is too far away from where they live (M1.35). Although the area in which the health house is located is actually very close to the city center, it seems almost deserted (very few people were observed outside the health house and only a few cyclists on the cycle path). It is a relatively new residential area, and there are (still) only a few things there besides housing (and the health house). Put differently, it is not really a place people would drop by (unless they live there, and those who do live there do not require local health activities). And if you do not know about the presence of the health house and where it is located, it is very difficult to find.
\par
To a certain extent, this deserted place with automatic lights and semantic instructions on the floor serves as an example of a physical script that nudges people into not talking or interacting in a particular manner. The organization of the waiting room scripts people to sit still and wait, and even though the café can be interpreted as an attempt to soften this, the machine and a few tables in an empty room still makes the room seem sterile and makes the interaction style highly ritualized and controlled towards a passive performance. Even though the spatial surroundings are different from a hospital, the nudging of a passive interaction style resembles an authoritarian relation between a doctor and his patient, as e.g. described by Parson (Parsons, 1975). This is also emphasized by the ‘push-button’ reception, which the staff insists on, even in situations where there is no real need for systematic ‘people-processing’, i.e. in situations where there are no other patients waiting and only employees present in the health house. During the observation, an old lady enters the house and when she is told by an employee how to push the button to access, she comments the button like this: “this is certainly very modern”. Her reaction demonstrates a small protest by highlighting a difference between common standards of interaction and, according to her, the health house’s over-ritualized way (“very-modern”) of welcoming her. This way of welcoming works as a disciplinary tool, teaching her that she is not more important than other citizens and that she is only one among many clients who require the service of the providing professionals.

\section{Health Professionals in a Medical--Clinical Health House: “We Are Health Consultants”}
The following analysis describes a health professional trained as a nurse (M1.35), who sees herself as a consultant for citizens with a focus on diet and exercise, but also as someone who motivates people to become ‘change consultants’ in their own lives. Her relation to the medical field structures many of her narratives about herself as a health professional, even though she has no medically trained colleagues. In the beginning of the interview, she explains her experiences with the dominance of doctors in the health field. She explains that she perceives the doctor as the best coordinator of the health professionals’ work with citizens: ‘the doctor is a kind of coordinator between us and the hospital’. By saying this, she gives the doctor an important role in her professional work, in spite of no formal or social relationships between her and the medical profession beyond the referral. Even though she distinguishes herself from a classic medical professional, she still emphasizes the monopoly of medical knowledge as a legitimate power bound to the hospital’s way of organizing intervention in relation to citizens. This distinction between the role of a hospital nurse and her own professional identity as a health consultant coupled with her faith in medical knowledge, splits her in a way that is simultaneously destructive and beneficial, and precisely what gives her a sense of professional direction in her job at the health house.
\par
To her, part of being a health consultant implies a professional dimension of coordination. In practice, this has to do with nurturing the relations with other public, private and NGO organizations, which she explains as follows:
    \blockquote{I am the person who coordinates and organizes our relations with the municipality and the entire civic society, and I find out what kind of knowledge we need to share. (M1.35)}
\par
She emphasizes how the intervention in the health house is not dependent on a particular diagnosis or treatment, but instead focused on the citizen’s potential for change and lifestyle improvements an area where doctors and the medical field have no monopoly or specialized expertise. In a way she refers to being sick as a state of being that also contain social aspects (Turner, 2007). Even in the absence of an actual medical professional, she emphasizes the doctor as the legitimate ‘organizer’ of the intervention. She appears to be normatively scripted to acknowledge the doctor as the ‘master’ and ‘natural coordinator’ of her intervention. However, during the interview it becomes clear that she is ambivalent about her status as a health professional and she swings back and forth between acknowledging doctors as the proper coordinators and making a counter-identification with them by seeing them as opponents to her professional expertise. As an example of this ambivalence, and a sign of opposition towards the medical profession, she tells a story about a doctor who called her because she did not follow his instructions to send a citizen to a smoking cessation course:
    \blockquote{A doctor once phoned me as he wanted to report me to the Board of Health because he had referred a citizen to a smoking cessation course, and I hadn’t done it, but instead had him participate in a different course called ‘Life Change’ where they did not work with smoking cessation (…). But, yes, I think if I can get a good relationship with him and make him trust me, then it may well be that, at some point, he will talk about the smoking with me. He is not stupid. He knows very well that it makes him uncomfortable and ill. Yes. (M1.35).}
Here, she not only reveals a conflict with the doctor over who has the authority to define the citizens’ problems, she also explains why she is not willing to comply with traditional medical conventions in her interventions. On a higher level, she associates health problems and rehabilitation with a relation between health professionalism, sickness and society, which is different from the notion of disease in the medical profession.
\par
This story explains why the health consultant does not accept dominance of medical professionals. She does not acknowledge the GP’s judgment as valid in this matter, because according to her the core issue is not treatment, which is the dominant approach in doctors’ health activities, but the mindset of the citizen and her relation to him. This counter-identification with traditional medically based health professionalism is found as a general perception across the interviews, though most significantly among health professionals from medically-clinically based health houses. Health professionals feel provoked and intimidated by the medical profession and at the same time dependent on it. They need the medical profession to provide them status as health knowledgeable in encounters with citizens, and at the same time they need professional autonomy to be ‘real’ health consultants in the health house.
\par
The conflict is also noticeable when, during the interview, she is exposed to vignette B, showing a medical-clinical health house very similar to her own. In her response, she positions the interdisciplinary activities of the health house against the mono-disciplinary activities in the hospitals. In contrast to her expressions in the beginning of the interview, she eventually outlines quite clearly her professional identity as being different from that of an assistant role to the doctor (M1.35). It is as though the exposure of a health house that resembles her own makes her see the boundary to the medical profession in a different and more contrasting light. This, of course, may also be due to a probing effect from the vignette, since before being exposed to the vignettes her response to questions about organizational relations and cooperation with other agents was to emphasize a subordinate dependence on the hospital. When exposed to it, she changes her statements regarding her ‘medically subordinated role’ as a nurse, as she compares the two types of health houses described in vignette A and B respectively. She ends up, as do most of the health professionals, preferring the community-based health house described in vignette A to the medically/clinically based health house described in vignette B. Her main reason is that she can fulfill a more professional role as a health professional when she meets citizens in more community-based spatial surroundings. This suggests that the physical frames introduced to her in the vignettes gives her an idea of a different way to provide health service and a different, maybe more autonomous way to meet citizens, which lead her to counter-identify with the classic doctor-subordinated role of the nurse working within the physical settings and regulations of the hospital.
\par
Shortly after the interview with the health consultant, a citizen walks by and agrees to participate in an interview. This citizen is a young (early twenties) former cancer patient, who is now in rehabilitation.

\section{Former Cancer Patient, Now in Rehabilitation: “I Meet People Facing the Same Struggles as Myself”}
The woman is recovering from cancer and tells of how she is tired from fighting the disease, and pleased to come to the health house to rehabilitate, train and meet other people (M1.34). Coming to the health house is new to her. She has only been here once before. She is going to start an education as an occupational therapist but is on temporary sick leave. Even though she is in physical settings representing the more medically-clinically based health house, she prefers the community-based health house, described in vignette A: “It's much better. Much better socially, and it has a bit of culture too. Vignette B seems way to clinic-like for my taste” (M1.34). 
\par
To her, rehabilitation is related to occupational professionalism, though with a focus on exercise and ‘togetherness’. She experiences rehabilitation as being very different from the activities in the hospital, and she finds it ‘natural’ to be rehabilitated by the health professionals in the health house. In this way, she does accept this new type of health professional, even though she is very clear about her need to seek ‘more’ occupational knowledge and hence a more well-defined service, by which she means doctors at the hospital. Like the health professional, this citizen distinguishes the occupational expertise and service supplied in the health house from the expertise supplied by doctors at the hospital. She emphasizes how it is the results of the health house professionals’ concrete health activity and not the hospital that is currently helping her to overcome her cancer (M1.34). Therefore, this woman also sees the health house and the encounters with health professionals as highly effective in relation to her cancer rehabilitation. She meets up with 7-8 other people for training twice a week. In addition, she also sees her health house visits as a kind of local social service, because she meets others who are struggling with cancer like herself. She links the social dimension of her rehabilitation directly to recovery, and this stimulates both her engagement in the health house and her expectations to the service that she will receive in the health house.
\par
Even though health house M1, as characterized above, shares some basic characteristics with a hospital, such as the passive environment around the reception, it is different from a hospital in terms of professional composition and service delivery. However, the other main type characterized as the community-based health house is even more different from a hospital, as will be illustrated in the following description of M2. 

\section{The Community--Based Health House: “A Flat Screen Promotes Various Events”}
The following description is based on on-site observation of M2 (field note M2). This health house is located in a large building together with a department of the municipality’s midwifery practice, a sports kindergarten, some meeting rooms and a large sports hall. The building is located in an immigrant neighborhood on the outskirts of the city and is surrounded by a lawn. There is an outdoor basketball court, a barbecue area and a playground belonging to the sports kindergarten and also an integrated part of the building. There are a lot of shouting children playing in the playground.
\par
Entry is through glass doors. Once you enter, you find yourself on a kind of balcony. To the left is a café, where you can buy food and drinks. There are tables and chairs, where you can sit and eat. Next to the café, there is a coffee maker and a soda machine. The midwife practice, the bathrooms and the health house are located to the right. There is a small waiting area in the front consisting of a black leather sofa and a table with some chairs. There are also some stands with brochures and advertisements for various events, such as walking for women, football for adult men every Tuesday and a playroom for children on Saturdays. A hallway leads off to some offices. From the ‘balcony’, a large staircase leads down to a sports hall. The walls of the house are painted in bright colors like orange and yellow. There are plants all around, some abstract paintings in bright orange and yellow colors and Arabic writing on the walls. There is a flat screen, advertising various events.
\par
It is still early in the morning when the observer arrives at the health house. A café employee, a smiling man in his mid-40s with an ethnic minority background, is making a birthday cake shaped like a racing car. Another health house staff of different ethnic background walks around the house talking with colleagues from the café in a language other than Danish. The house is quiet, until a large group of kindergarten children (15--20) run into the sports hall. They shout and laugh. They are followed by five adults from the kindergarten (educators or assistants)---one man (about 30 years old) and four women (20--40 years old). Two of the women seem to be of Danish origin. The other two appear to have a different ethnic background. The educators shout at the children and ask them to sit down on the staircase, where they are given water and food from a table on wheels. Most of the children seem to be of ethnic minority background. Afterwards, the children play in the hall. They run around, laughing and shouting. During the observation, a number of people come and go. They talk and drink coffee, and only some of them seek contact with the health house staff. The café represents the entrance to the health house.
\par
Everyone in the building seems to know each other, and the place seems to be used a lot by the local citizens. There is a community center atmosphere in the main building, but also in the actual health house. The interactional styles appear to be much less ritualized compared to the atmosphere in the medically-clinically based house (M1). This, however, also made it difficult for the observer to feel comfortable and ‘part of the interactions’, which was never the case during the observation in the medically-clinically based house.
\par
When on-site observations between the medical-clinical and the community-based health houses are compared, the interaction between health professionals and the citizens in the community-based house appears to be less ritualized than in the medical-clinical setting, where citizens enter, wait, receive a service and leave the health house immediately after the encounter (M1 and M2). In the community-based house, citizens are encouraged by the physical settings to stay and talk to each other and to the health professionals. They do not perform on a specific scenic front, and the actions of the staff do not seem to be as ritualized as those of the staff in the medically-clinically based health house. Here, citizens sometimes enter just to have a coffee, or to promote local activities. It is in this respect that this house resembles more closely an assembly house in contrast to the medical-clinical based health house, which to a certain extent mirrors a hospitals’ physical and social setting. 

\section{Health Professional in a Community-Based Health House: “I Miss My Work When I Am on Vacation”}
After the observation, a health consultant was interviewed. She is a woman in her mid-thirties and presents herself as a dietitian (M2.32). It is very important to her to stress that she and her colleagues are highly professional, as though someone had questioned this. She comes from Iran. She teaches diet, exercise and stress management, and uses pedagogical principles to motivate citizens’ habit changes. During the interview, she says (more than once) that she loves her work and misses it when she is away on vacation. She also tells of how she occasionally cries with the citizens, because she gets very involved in some of their very difficult situations. In contrast to the health professional interviewed in the medically/clinically based health house, who emphasized her occupational competences, she sees patience as a virtue over specialized knowledge, and she sees it as her role to strongly identify with the citizens’ terms and potential for ‘life’. To stress this she says: ‘If they [citizens who are late] don’t know the time and do not show up at the appointed time, there is no point in getting angry – then you have the wrong job’ (M2.32). To her, the health house is a public setting more that it is a medical setting, and therefore she does not see it as her task to force citizens to go there or to alter their behavior. 
\par
The health professionals encounter both women and men between 30 and 70 years of age who are typically relatively disadvantaged financially. Many of them come because of the free training and exercise services. They do not speak Danish – those who do, have to use the medically-clinically based health house in another part of the city (M1 analyzed above). The health house provides interpreters, who are always available. In general, the citizens who come here are in a lot of pain, and they are typically on sick leave or retired. Many of them are illiterate and struggling with social problems and stress. Therefore, as M2.32 explains, she will have to deal with all these other issues before she can start focusing on their diet, which is her professional occupational focus. Many citizens have been tortured and have lost family members. 
\par
This health professional sees the purpose of the work in the health house as helping immigrants to a healthier lifestyle with COPD, diabetes and cardiovascular disorders, as well as stress and chronic pain. She also prefers the community-based physical setting described in vignette A, because to her, health is not only something physical. There are many issues at stake; family, economy, housing and the physical and mental environment around people: “If you involve all aspects, you are better able to handle the problems in society than if you just focus on one little thing”, as she explains. To her, creating a safe environment where citizens can get help for whatever they are struggling with is the primary value in her job. Summed up, she appears to have a professional identity which consists of strong personal, organizational and occupational dimensions (M2.32). In contrast to the health professional analyzed above, she makes no counter-identification with the medical profession. Instead, she points at particular personal qualities to explain her professionalism, such as patience and responsiveness towards citizens.

\section{Citizens on the Edge of the City: “It's the Best the Municipality Has Ever Done for Its Citizens”}
One of the citizens interviewed in M2 (M2.30) does not understand Danish very well (the interview is conducted with an interpreter), although he has lived in Denmark for over 20 years. He has children, and he has been on sick leave for three years, due to his body being worn out after years of factory work. Even though there is access to M2 without referral many of the citizens have been referred by their social worker or GP, as is also the case here. His caseworker referred him to M2. First, he was referred to participate in a diabetes course, and next in a pain-management course. 
\par
His reaction to the vignettes is interpreted in light of him explaining a need to have someone help him get his letters from the public authorities translated into Turkish. He prefers a health house which allows access without referral, and a health house that is also used by Danes rather than minority groups only. The latter is a reason why he likes the medical-clinical health house, described in vignette B (M2.30). This house has a fitness center, and he believes that Danes are more likely to go there than to the community-based house described in vignette A. However, very much in line with the majority of the interviewed citizens, he does not like the fact that access is through doctors only in vignette B: ‘I would like to combine the doctor from vignette B with the free access from vignette A’ (M2.30). This citizen talks about how he feels secure in his health house, and he even links social health to physical health, when he says that he is better able to live with his pain because of his engagement in the health house. Here, he never feels pressured by others, as he does in almost every other corner of Danish society (M2.30). To him the health house is a safe zone, where he can take a break from what he sees as a society that puts a lot of social pressure on him. He stresses how this is linked to the fact that he is allowed to work with himself and his health in his own language, i.e. there is not (as in virtually all other meetings with the Danish community), a Danish-ethnic agenda. Here, he can describe his diabetes and chronic pain in his mother tongue. Therefore, he sees no disadvantages in vignette A, apart from the fact that there is no fitness center. The house is more or less similar to his own health house.
\par
This citizen has experienced falling outside the municipal and regional aid systems, and he blames his doctor and caseworker for this exclusion. He sees the doctor and the caseworker as incapable of cooperating and coordinating an effort across organizational boundaries. Interpreted in light of Evetts idea of hybrid professionalism, his perceptions may be said to reflect what hybrid professionalism looks like from the citizen perspective. He needs a caseworker to help him not only with access to service, but also with communication to other authorities and he needs a doctor to treat him and to provide access to other health activities, but also to coordinate health promotion and physical training with other health professionals. These skills include general organizational competences, which he feels the health professionals in the health house have but not the caseworker and the doctor in his story. He also associates the value of the health house to the fact that access is not governed by certain public authorities and hereby emphasize the organizational autonomy of the health house as an advantage even for him as a patient (M2.30). He sees the health house as a place where he is able to learn to live with his illness on his own terms. Like other citizens, he uses the health house to exercise frequently, something he would never do in a regular fitness studio or at the hospital. More specifically, he explains that the yield he gets from going to the health house is that he now has something to wake up to in the morning. His self-efficacy seems to have increased through his engagement in the health house and his capacity to act in relation to his suffering. He can now handle his pain to an extent that allows him to seek pleasures in life. He repeatedly emphasizes the social aspect of the health house and how gaining knowledge with others makes a positive difference in his life. To him, the health house is a real alternative to the hospital, but also a safe zone from a demanding society (M2.30).

\chapter{Conclusion: Counter-Identification With Hospitals and Doctors}
The health professionals working in the health houses offered citizens a wide range of health activities. They were typically related to exercise, life-style education, social activities, rehabilitation and diet. In addition to what can be termed ‘occupational services’, such as physiotherapy or diet consultancy, they also served other organizational functions, such as communicating with different divisions in the municipality, engagement with private actors about in-house activities and ‘touring’ around GP practices to inform about the health house and what kind of local health activities they offer citizens.
\par
Generally, health professionals tried to differentiate their services from what is offered in hospitals. In terms of processes of identification this means that even in the absence of doctors, medical categories influenced how they thought about themselves. Some counter-identified themselves and the health house activities with medical doctors and hospitals by promoting health house activities as being more comprehensive than hospital services. One explained how the activities was invented and made possible because of the organizational setting of the health house. Here, the analysis showed that the health houses’ physical frame and organizational setting of the encounters with citizens differed. The health professional working in the medically-clinically based health house was more influenced by medical categories for describing health and by relations to the medical field compared to the health professional working in the community-based health house. The health professional working in the medically-clinically based health house wants to be a community-based health professional, but she does not describe herself as one, precisely because she is embedded in a medically-clinically based frame. Another dimension of this difference was apparent in understandings of health activities as best coordinated by doctors, as in the medically-clinically based health house, or to resilience, as in the community-based health house. The spatial surroundings in the community-based setting made both the health professional and the citizen reflect and act differently from those situated in the medical-clinical setting.
\par
The emergence of local health houses seems to provide the organizational basis for community-based health professionalism. The analysis of spatial surroundings in the health houses shows how different forms of old and new relations between health professionals, citizens and society exist at the same time. In the analysis, relations between the health professional and the citizen in the medically-clinically-based surrounding draw more on medical categories of treatment, whereas relations between the health professional and the citizen in the community-based surrounding indicated a more civic engaging style. For example this was expressed as “It's the best the municipality has ever done for its citizens” (M2.30).
\par
Both the citizens and the health professionals preferred a health house that is community-based with free access to activities. However, in contrast to the health professionals, who see the community-based setting as better for positive health professional work, citizens prefer the free access because they miss health-related professional attention. To some extent, this suggests that citizens expect to meet a doctor with diagnostic skills and not the health professional that they meet in the health house. This finding speaks to existing work of contested professions with a focus on jurisdiction and managerial terrain and what ways are available to new professions trying to claim social and cultural authority (Abbott 1988: 89-90).
\par
In addition to the question of what paves the way for new professions, the analysis also speaks to the study of hybrid professionalism and its dynamic relation to citizens and society. Health professionals’ relations to citizens were not deduced from mindsets given by their professional backgrounds, but were shaped and interpreted in light of the options for interactions given by the spatial surroundings of the health house. Not surprisingly, these spatial surroundings were supported with the overall organizational structure such as to what extent gate-keeping was a principle for how to manage citizens and to what extent rehabilitation or health promotion framed the policy of encounters with citizens in the particular health house. However, the vignette responses revealed a clear preference for a particular organization and spatial surrounding, which indicate that neither administrative scripts nor professionalisms are determinant for how health professionals view health, define problems, perceive of citizens or how they perform.
\par
The health houses can be said to express a new relationship between health professionals, citizens and society. In the health houses, health professionals encounter individuals as citizens and not as patients. Citizens who visit a health house expect to meet a doctor with diagnostic skills, and the relationship between health professionals and citizens might be challenged by the fact that health professionals cannot expect citizens to respect them as health experts, because they are not medical doctors, but only ‘semi-professionals’ (Etzioni, 1969). In other words, they risk losing their expert status in the encounter with citizens, who expect to meet a full-blooded medical doctor. Health professionals in health houses represent a new kind of local health professionalism, and they need to define their role in relation to citizens. Following theory on medical sociology, their ability to define their own room of maneuver also depends on control over a physical setting that belongs to them, just as the clinic belongs to the medical profession (Foucault, 2000).
\par
Even though some degree of self-selection did take place, the strict process of zooming in on two health houses representing significant differences within the large sample does ensure that the material was saturated on the dimensions analyzed here. Therefore, despite the fact that this analysis does not allow for empirical generalization beyond the sample these two health houses were selected from, it may be applied analytically to other organizations with relatively low occupational gatekeeping and a relatively diverse composition of professionalism. \par
Finally, the analysis sheds light on how macro structures of organizational ideas, physical settings and professionalization are visible in the microcosm of social interactions such as the ones under study here. More specifically, Goffman’s claim that spatial surroundings facilitate processes of identification, that is how frontline desks, information screens and coffee tables not only design houses, but also script styles of interaction between individuals as well as they create expectations about how to reflect and act in a given physical and social setting became visible in the observations and the interviews. Here, the analysis points at a rather complex condition for how the processes of identification takes place: On one hand both citizens and health professionals understand themselves and their role in the health house through an identification or counter-identification with what hospitals and doctors’ do. On the other hand this relationship between a stable and dominant frame of reference and the experience of a new professional terrain appears as interwoven with the ‘new’ spatial surroundings of the houses.