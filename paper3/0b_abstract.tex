    \begin{adjustwidth}{20mm}{20mm}
\label{paper3:abstract}
    \bigskip
    \begin{otherlanguage}{english}
    {\small
%    \fadebreak
%%%%%%%%%%%%%%%%%%%%%%%%%%%%%%%%%%%%%%%%%%%%%%%%%%%%%%%%
%%%%%%%%%%%%%%%% START ABSTRACT.tex %%%%%%%%%%%%%%%%%%%%
%%%%%%%%%%%%%%%%%%%%%%%%%%%%%%%%%%%%%%%%%%%%%%%%%%%%%%%%

\noindent The article uses the organization of health houses in Denmark as a case to study the relationship between spatial surroundings and professionalization. The question is whether these new local health houses comprise an alternative to the medical view on health or---even in the absence of the hospital---script the professionals to identify themselves as agents from the medical field? In this article, macro-structural theory is combined with micro-relational theory in order to identify how macro structures such as professionalization nest the way social interaction takes place in concrete spatial situations and surroundings. The argument put forward is that we need to identity this process at the level of the individual in order to qualify and anchor our understanding of professionalization as a macro phenomenon. The empirical basis is two dissimilar locations (health houses), selected from a larger qualitative data set of interviews with health professionals and citizens and observations of health houses, originally selected from a nationwide survey. The presented analysis zooms in on selected places and situations and relates analyses to the overall picture of differences and similarities identified in the larger sample. The analysis shows how entrances, receptions, information screens and coffee tables not only design houses, but also script styles of interaction between health professionals and citizens as well as they work as signs creating expectations about professional roles and how to reflect and act as a professional in a given physical and social setting. The main finding is that spatial surroundings facilitate processes of identification and counter-identification crucial to a new kind of health professionals such as the ones under study here.

\smallskip
\noindent\rule{\linewidth}{1pt}

%%%%% KEYWORDS
%
\noindent{\bfseries Keywords:}\hspace*{0.75em}{% list keywords below
healthcare professionalism;
spatial surroundings;
counter-identification;
health house service;
vignettes}.

    } % ends font family

%    \fadebreak

    \end{otherlanguage}

    \end{adjustwidth}