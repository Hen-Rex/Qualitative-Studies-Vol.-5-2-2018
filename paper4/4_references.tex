\label{paper4:references}
\begin{thebibliography}{99}
%%%%%%%%%%%%%%%%%%%%%%%%%%%%%%%%%%%%%%%%%%%%%%%%%%%%%%%%
%%%%%%%%%%%%%%% START 9_references.tex %%%%%%%%%%%%%%%%%
%%%%%%%%%%%%%%%%%%%%%%%%%%%%%%%%%%%%%%%%%%%%%%%%%%%%%%%%

\bibitem{1} Arendt, Hannah. 1963. \textit{Eichmann in Jerusalem: A Report on the Banality of Evil}. Viking Press.
\bibitem{2} Bauman, Zygmunt. 2008. \textit{The Art of Life}. Cambridge (UK); Malden: Polity Press.
\bibitem{3} Beck. 2008. \textit{World at Risk}. Oxford: Polity.
\bibitem{4} Blau, Peter Michael, and Marshall W. Meyer. 1987. \textit{Bureaucracy in Modern Society}. McGraw-Hill.
\bibitem{5} Borrelli, Lisa Marie, 2018. ‘Whisper down, up an between the Lane – Exclusionary Policies and Their Limits of Control in Times of Irregular Migration’, \textit{Public Administration}, 1-14. DOI: 10.1111/padm.12528.
\bibitem{6} Borrelli, Lisa Marie, and Annika Lindberg, 2018. ‘The Creativity of Coping: Alternative Tales of Moral Dilemmas among Migration Control Officers.’, \textit{International Journal of Migration and Border Studies} 4(3): 163-178. DOI: 10.1504/IJMBS.2018.10013558.
\bibitem{7} Chauvin, Sébastien, and Blanca Garcés-Mascareñas. 2014. ‘Becoming Less Illegal: Deservingness Frames and Undocumented Migrant Incorporation: Becoming Less Illegal’. \textit{Sociology Compass} 8 (4): 422–32. Doi: 10.1111/soc4.12145.
\bibitem{8} Cohen, Stanley. 2001. \textit{States of Denial: Knowing about Atrocities and Suffering}. Wiley.
\bibitem{9} Croissant, Jennifer L. 2014. ‘Agnotology: Ignorance and Absence or Towards a Sociology of Things That Aren’t There’. \textit{Social Epistemology}, 28 (1): 4–25. Doi: 10.1080/02691728.2013.862880.
\bibitem{10} Deleuze, Gilles. 1994. \textit{Difference and Repetition}. New York: Columbia University Press.
\bibitem{11} Douglas, Mary. 1986. \textit{How Institutions Think}. 1st ed. The Frank W. Abrams Lectures. Syracuse, N.Y: Syracuse University Press.
\bibitem{12} Eule, Tobias G. 2014. \textit{Inside Immgration Law. Research in Migration and Ethnic Relations Series}. Farnham, UK: Ashgate Publishing Limited.
\bibitem{13} Eule, Tobias G., Lisa Marie Borrelli, Annika Lindberg, and Anna Wyss, 2018. \textit{Migrants Before the Law: Contested Migration Control in Europe}. Palgrave Macmillan.
\bibitem{14} Ewick, Patricia, and Susan S. Silbey. 1995. ‘Subversive Stories and Hegemonic Tales: Toward a Sociology of Narrative’, \textit{Law \& Society Review} 29(2): 197–226.
\bibitem{15} Fassin, Didier. 2013. \textit{Enforcing Order: An Ethnography of Urban Policing}. Cambridge: Polity Press.
\bibitem{16} Feldman, Gregory. 2012. \textit{The Migration Apparatus: Security, Labor, and Policymaking in the European Union}. Stanford, California: Stanford University Press.
\bibitem{17} Flick, Uwe. 2011. \textit{Triangulation: Eine Einführung}. 3rd ed. Qualitative Sozialforschung. VS Verlag für Sozialwissenschaften \url{https://www.springer.com/de/book/9783531181257}.
\bibitem{18} Frye, Marilyn. 1983. ‘On Being White: Thinking Toward a Feminist Understanding of Race and Race Supremacy’. \textit{The Politics Of Reality: Essays In Feminist Theory.}
\bibitem{19} Galison, Peter. 2004. ‘Removing Knowledge’. \textit{Critical Inquiry} 31 (1): 229–43. Doi: 10.1086/427309.
\bibitem{20} Garver, Newton. 1973. ‘What Violence Is’. \textit{In Philosophy for a New Generation}., edited by A.K. Bierman and James A. Gould, 2nd ed, 256–66. New York: Macmillan.
\bibitem{21} Geertz, C. 1973. Person, Time, and Conduct in Bali in His ‘\textit{The Interpretation of Cultures}’. New York: Basic Books.
\bibitem{22} Giddens, Anthony. 1979. \textit{Central Problems in Social Theory: Action, Structure and Contradiction in Social Analysis}. Berkeley: University of California Press.
\bibitem{23} Gilliom, John. 2001. \textit{Overseers of the Poor: Surveillance, Resistance, and the Limits of Privacy}. Chicago, IL: University of Chicago Press.
\bibitem{24} Graaf, Gjalt de, Leo Huberts, and Remco Smulders. 2014. ‘Coping With Public Value Conflicts’. \textit{Administration \& Society }48 (9): 1101–27. Doi: 10.1177/0095399714532273.
\bibitem{25} Gupta, Akhil. 2012. \textit{Red Tape: Bureaucracy, Structural Violence, and Poverty in India}. A John Hope Franklin Center Book. Durham: Duke University Press.
\bibitem{26} Herzfeld, Michel. 1992. \textit{The Social Production of Indifference}. Chicago: University Of Chicago Press.
\bibitem{27} Høivik, Tord. 1977. ‘The Demography of Structural Violence, The Demography of Structural Violence’. \textit{Journal of Peace Research }14 (1): 59–73. Doi: 10.1177/002234337701400104.
\bibitem{28} Jackson, Michael. 2013. \textit{The Politics of Storytelling. Variations on a Theme by Hannah Arendt}. Chicago: University Of Chicago Press.
\bibitem{29} Galtung, Johan. 1969. ‘Violence, Peace, and Peace Research’. \textit{Journal of Peace Research} 6 (3): 167–91.
\bibitem{30} Lipsky, Michael. 2010. \textit{Street-Level Bureaucracy: The Dilemmas of the Individual in Public Service}. New York: Russell Sage Foundation.
\bibitem{31} McGoey, L. 2012a. ‘Strategic Unknowns: Towards a Sociology of Ignorance’. \textit{Economy and Society}, 41 (1): 1–16.
\bibitem{32} McGoey. 2012b. ‘The Logic of Strategic Ignorance’. The British \textit{Journal of Sociology}, 63 (3): 553–76.
\bibitem{33} Michaels, D. 2008. ‘Manufactured Uncertainty. Contested Science and the Protection of the Public’s Health and Environment’. In \textit{Agnotology. The Making and Unmaking of Ignorance}., edited by Robert N. Proctor and Londa Schiebinger, 90–107. Stanford, California: Stanford University Press.
\bibitem{34} Moore, W. E., and M. M. Tumin. 1949. ‘Some Social Functions of Ignorance’. \textit{American Sociological Review,} 14 (6): 787–95.
\bibitem{35} Nair, Rukmini Bhaya. 1999. ‘Postcoloniality and the Matrix of Indifference’. \textit{India International Centre Quarterly} 26 (2): 7–24.
\bibitem{36} Ortner, Sherry B. 2006. \textit{Anthropology and Social Theory: Culture, Power, and the Acting Subject}. Duke University Press. \url{https://doi.org/10.1215/9780822388456}.
\bibitem{37} Proctor, Robert N. 2008. ‘Agnotology. A Missing Term to Describe the Cultural Production of Ignorance (and Its Study)’. In \textit{Agnotology. The Making and Unmaking of Ignorance}., edited by Robert N. Proctor and Londa Schiebinger, 1–33. Stanford, California: Stanford University Press.
\bibitem{38} Rayner, S. 2012. ‘Uncomfortable Knowledge: The Social Construction of Ignorance in Science and Environmental Policy Discourse’. \textit{Economy and Society}, 41 (1): 107–25.
\bibitem{39} Rylko-Bauer, Barbara, and Paul Farmer. 2016. ‘Structural Violence, Poverty, and Social Suffering’. In \textit{The Oxford Handbook of the Social Science of Poverty}, edited by David Brady and Linda M. Burton. Oxford: Oxford University Press. 
\bibitem{40} Scott, James C. 1990. \textit{Domination and the Arts of Resistance: Hidden Transcripts}. New Haven and London: Yale University Press.
\bibitem{41} Sedgwick, Eve Kosofsky. 1990.\textit{ Epistemology of the Closet}. Berkeley: University of California Press.
\bibitem{42} Shaw, George Bernard. 2015. \textit{The Devil’s Disciple}. Project Gutenberg EBook. \url{https://www.gutenberg.org/files/3638/3638-h/3638-h.htm}.
\bibitem{43} Simmons, William Paul, and Monica J. Casper. 2012. ‘Culpability, Social Triage, and Structural Violence in the Aftermath of Katrina’. \textit{Perspectives on Politics} 10 (3): 675–86. Doi: 10.1017/S1537592712001697.
\bibitem{44} Slater, T. 2012. ‘The Myth of ”Broken Britain”: Welfare Reform and the Production of Ignorance’. \textit{Antipode}, 46 (4): 948–69.
\bibitem{45} Smithson, Michael.  1989. \textit{Ignorance and Uncertainty}. Cognitive Science. New York, NY: Springer New York. DOI: 10.1007/978-1-4612-3628-3.
\bibitem{46} Smithson, Michael. 2008. ‘Social Theories of Ignorance’. In \textit{Agnotology. The Making and Unmaking of Ignorance}., edited by R. N. Proctor and L. Schiebinger, 209–19. Stanford, California: Stanford University Press.
\bibitem{47} Smithson, Michael. 2010. ‘Understanding Uncertainty’. In \textit{Dealing with Uncertainties in Policing Serious Crime}, edited by Gabriele Bammer, 27–48. ANU Press. 
\bibitem{48} Stel, Nora. 2016. ‘The Agnotology of Eviction in South Lebanon’s Palestinian Gatherings: How Institutional Ambiguity and Deliberate Ignorance Shape Sensitive Spaces: The Agnotology of Eviction’. \textit{Antipode} 48 (5): 1400–1419. Doi: 10.1111/anti.12252.
\bibitem{49} Sykes, G. M., and D. Matza. 1957. ‘Techniques of Neutralization: A Theory of Delinquency’. \textit{American Sociological Review}, 22 (6): 664–70.
\bibitem{50} Thompson, Dennis F. 1980. ‘Moral Responsibility of Public Officials: The Problem of Many Hands’. \textit{American Political Science Review} 74 (4): 905–16. Doi: 10.2307/1954312.
\bibitem{51} Tuana, Nancy. 2008. ‘Coming to Understand: Orgasm and the Epistemology of Ignorance’. In \textit{Agnotology. The Making and Unmaking of Ignorance}., edited by Robert N. Proctor and Londa Schiebinger, 108–48. Stanford, California: Stanford University Press.
\bibitem{52} Tummers, Lars L. G., Victor Bekkers, Evelien Vink, and Michael Musheno. 2015. ‘Coping During Public Service Delivery: A Conceptualization and Systematic Review of the Literature’. \textit{JPART}, 25: 1099–1126. Doi:.1093/jopart/muu056.
\bibitem{53} Watkin, William. 2014. \textit{Agamben and Indifference: A Critical Overview}. London; New York: Rowman \& Littlefield International.

%##################################################################################################################################################################################################################################
\end{thebibliography}