%%%%%%%%%%%%%%%%%%%%%%%%%%%%%%%%%%%%%%%%%%%%%%%%%%%%%%%%
%%%%%%%%%%%%%%%%% START 1_ARTICLE.tex %%%%%%%%%%%%%%%%%%
%%%%%%%%%%%%%%%%%%%%%%%%%%%%%%%%%%%%%%%%%%%%%%%%%%%%%%%%

%\chapter{Introduction}
    \blockquote{I am taking part in a summons regarding the deportation of a family to Italy. The Swiss Cantonal Migration Officer F asks the father if they are willing to leave. He answers: ‘Would I be alone, yes, but I have a family. We do not want to live like animals. It does not suffice - food, portable toilets, no roof over our heads. And they are threatening us to deport us back to Lebanon.’ F does not react to this answer and continues working on the forms, in front of his laptop, skipping to the next question: ‘Do you have other documents than your identity cards?’ – ‘No.’ The situation is tense, the translator gets increasingly uncomfortable, one of the children is about to cry. A quick exchange on the poor mental and physical health of the family is followed by the father’s comment: ‘I am not responsible if something happens to my family. You are responsible.’ F continues filling in his forms, turns around to the translator, with a final remark and question: ‘Yes, sure. You are communicating in Arabic?’ (Swiss Cantonal Migration office 2016)}\columnbreak
\noindent\lettrine[lines=2]{\bfseries\color{black}B}{ureaucratic encounters}, such as the one above, have often been characterised as painstaking, absurd and chiefly as a violent interaction caused by bureaucratic indifference (Herzfeld 1992). Also recent and former fictional depictions of bureaucracy (see Loach 2016, Kafka’s ‘The Trial’ 1925 and Gogol’s ‘The Overcoat’ 1842) regularly take up the issue of a careless and user unfriendly, even hostile environment of public administration, elucidating what Gupta (2012) described as individuals’ - including the welfare ‘poor’ (cf. Gilliom 2001) or migrants\pagenote{This article wants to highlight that besides the seldomly ‘voluntary’ visit of individuals to the municipality or other state agencies, migrants are neither citizens nor ‘clients’ of welfare, and as such are not entitled to be treated as such. The encounter with bureaucrats is often forced upon these migrants with precarious legal status, on which this study focuses, and strongly impacts on migrants’ lives. Throughout this work I will refer to the term ‘migrant  with precarious legal status’ if generally talked about; otherwise I will refer to the respective legal status, such as detainee, Dublin deportee, rejected asylum seeker, and other.}  ---experiences of structural violence in encounters of public administration.
\par
The fictional and actual study of bureaucratic behaviour (cf. Lipsky 2010; Fassin 2013; Herzfeld 1992) mutually highlight a general disinterest and incomprehension on the side of the bureaucrat towards their clients. The arbitrary position towards the applicants’ needs is especially observable in forced encounters between migrants with precarious legal status and street-level bureaucrats, discussed in this work. For this marginalised group, bureaucratic interaction usually implies a negative outcome: detention and deportation. 
\par
However, the encounters are not only demarcated by indifference or disinterest, but also by ignorance. In order to understand the dynamics of these emotionally laden and contested encounters, disregarding whether the interaction happens willingly or is coerced, this work will introduce the theoretical concept of ignorance, which adds to theories on street-level bureaucracy. Ignorance as deliberate or undeliberate lack of or gaps in knowledge (in a simplified definition), contributes to our understanding of bureaucrat’s everyday practices within the migration regime as it reveals how (non-)knowledge can be (un)consciously used by all actors involved, including bureaucrats and migrants, as well as be an inherent part of the ‘state’ and its structures.
\par
This article thus analyses the creation and use of ignorance in government agencies dealing with the enforcement of migration policies on a structural and individual level. While it postulates that ignorance is an inherent and integral part of the otherwise non-unitary state structure, it steps further and aims to analyse the street-level bureaucrats’ use of ignorance, which includes un-knowing, partial knowing or blinding out knowledge. By analysing the intertwined relation between structural and individual ignorance, this work is able to contextualise uncertainty and lacking accountability as outcomes of daily bureaucratic encounters. These include the gap of knowledge on  procedures which either are not part of the daily work, or - having a greater impact -  should be known but are ignored. On a more moral level, it encompasses bureaucratic behaviour regarding the more personal side of the encounters, where empathy might be kept hidden, because of more personal reasons and pragmatism. It is thus able to not only raise relevant concerns about how administrative and moral aspects of ignorance create intangible practices and emotionally charged encounters, but also unsound practices.
\par
The analysis, looking at strategies of ignorance used by bureaucrats, migrants and within state structures, is preceded by a theoretical conceptualisation of ignorance in relation to indifference and uncertainty. It is followed by a brief methodological description of where and how fieldwork was conducted. Finally, the conclusion summarises the tactics used in the daily work of public administration. Distinguishing between different individual and structural ignorance facilitates a better comprehension of how structural violence, uncertainty and diffusion of accountability are reproduced, but also under what circumstances.

\chapter{Theoretical Framework}
When conceptualising discretionary spaces, several strategies such as coping mechanisms (de Graaf, Huberts, and Smulders 2014; Tummers, et al. 2015) , cherry-picking or foot dragging (Scott 1990; Lipsky 2010; Eule 2014) explain the realities of public offices. Further, routinisation and attitudes of indifference to detach oneself from work practices (Blau and Meyer 1987; Herzfeld 1992) and the diffusion of many hands (Thompson 1980) underline ways of bureaucrats dealing with their everyday work, reducing their accountability. With Tummers, et al. (2015) broadly conceptualise coping as ‘behavioural efforts frontline workers employ when interacting with clients, in order to master, tolerate, or reduce external and internal demands and conflicts they face on an everyday basis’ (Tummers, et al. 2015, 1100; cf. Borrelli and Lindberg, 2018, on creative strategies), ‘coping’ presupposes a conscious action. Ignorance in contrast does not only materialise in reactions to the cross pressure of policy demands, citizens’ claims and the state agencies’ organisation, or as practices reflecting personal ideas and values, which are actively pursued. Instead it  also encompasses more unconscious moments, where one is not aware of knowledge, or moments where one is held ignorant. Ignorance thus plays a crucial role in migrant-bureaucrat encounters, since it brings forwards how it can be used as coping mechanism, but also how it is structurally placed and unconsciously imposed, while influencing the everyday practices and reflections of street-level bureaucrats (as well as migrants).
\par
Ignorance can surely be found in other administrative contexts and thus applies to other bureaucrats, such as social workers or teachers. While the findings can be used in a broader context, this study however wants to highlight the particularity of the public administration studied here. Within the migration apparatus (Feldman 2012), bureaucrats work in a field of severe ostracism and coercive practices towards an already strongly marginalised group. In this apparatus they can either use ignorance as a tactic to enforce power-inequalities pre-established by legal frameworks, or subvert them. They are at the same time influenced and constrained by their knowledge and the gaps, created by the structural organisation of state agencies. In either way, the existence and use of ignorance, passive or active, subtle or crude, highlights the extreme situation migrants face in their everyday lives, as well as the state’s capacity to govern them.
\par
As such, the use of ignorance in context of irregularised migration highlights the reproduction of structural violence, and also reveals how ignorance that is structurally imposed on the bureaucrat supports the increasingly restrictive position of the researched countries towards migration (cf. Borrelli, 2018; Eule, et al., 2018). Indeed, similar to Herzfeld’s (1992, 1, 18f) and Arendt’s (1963, 283) take on how given bureaucratic structures create and amplify indifference and a banal evil among frontline staff, their legal mandates and tasks of the interlocutors, as well as the organisational structures facilitate ignorance ---something presented in the analytical section.        

\subsection{Defining Structural Violence in Bureaucratic Encounters}
In the following, structural violence will mainly be understood according to Galtung (1969) and Gupta (2012). While Galtung (1969) names unequal distribution of power and unequal life chances, caused by poverty, marginalisation and exploitation (Galtung 1969, 171; cf. Rylko-Bauer and Farmer 2016; Garver 1973), Gupta (2012, 20) adds the inability to identify a single actor responsible for a violent act to the concept of structural violence.
\par
Though this concept is mostly used when studying the loss of life due to social conditions (Høivik 1977; Simmons and Casper 2012), the depicted encounters between street-level bureaucrats and migrants with precarious legal status strongly reflect structural violence in a more banal way. Especially since irregularised migrants are a marginal group with fewer rights, though often contributing with work and tax payments to societies which reject their presence (Chauvin and Garcés-Mascareñas 2014), it is they who are systematically being denied agency (Jackson 2013) through the use of ignorance. Thus, combing the concepts of ignorance and structural violence will not only tell us how the former supports the latter, but also how ignorance legitimises structural violence as it limits the ‘other’s’ agency,  for example by consciously creating knowledge gaps and not acknowledging the other’s aspirations and thus, voice.  It highlights the inequalities that shape power relations between the migrant subject and the ‘state’, represented by the bureaucrat in the European migration regime. 

\subsection{The Relation Between Indifference and Ignorance}
    \blockquote{The worst sin towards our fellow creatures is not to hate them, but to be indifferent to them: that's the essence of inhumanity. (Shaw 2015)}
\noindent Following the Oxford dictionary, indifference is described as a ‘lack of interest, concern, or sympathy’, including a notion of unimportance. Connected to Nair’s (1999) understanding, indifference is the ‘language of denial’, ‘achieved in institutional set-ups where bureaucratic rules end up thwarting, even damaging, every people they were meant to help’ (ibid.: 13). However, bureaucrats do not simply treat each case similarly, thus suggesting indifference can be characterised by an absence of feelings (Watkin 2014, 50), but bring their attitudes into the processing of cases.
\par
This stands in contrast to indifference as absence of meaning or relationship (Deleuze 1994). Indifference is a disinterest towards the person, which does not motivate the bureaucrat to familiarise themselves with the client’s case file or history. Ignorance can instead be influenced by (unconscious) knowledge gaps, personal values and an active manipulation of information. While both indifference and ignorance can help the bureaucrats to not step beyond their actual tasks, ignorance goes beyond the concept of ‘not caring’ (Herzfeld 1992) and can function as moral resistance (Proctor 2008). Although both, ignorance and indifference often refer to denial (Nair 1999) and are immanent in the space of bureaucracy, both terms have to be seen as related but not equal concepts. 

\subsection{Understanding Uncertainty}

Besides indifference being partly connected to ignorance, uncertainty also plays a crucial role in understanding how structural violence is produced and upheld. This work argues that besides the general uncertainty existing in migrants’ everyday life, it is strongly produced through ignorance during bureaucratic encounters. Especially in context of deportation and detention, where power struggles and unequal forces are tangible, uncertainty is a serious outcome for the migrants, who are held in a status of ‘not knowing’ or imperfect knowledge (Smithson 2010).  Uncertainty is linked to an outcome which lies in the future, characterised by delay and distance (cf. Beck 2008), although its consequences are fairly palpable (e.g. anxiety, stress).
\par
This differentiation is crucial with regards to the outcome of ignorance. The latter is able to maintain and manipulate behaviour, knowledge transfer and information, resulting in uncertainty as a mode of being kept ignorant and manipulated (Proctor 2008). Thus, ignorance encompasses gaps of knowledge and forms of resistance (ibid., 8), while actively or passively producing uncertainty. 

\subsection{Understanding Ignorance}
Ignorance has been defined in various ways in scholarly literature (cf. Smithson 1989; Galison 2004). While individuals do not have the capacity to know everything  (Douglas 1986) and do not have access to all knowledge, they also have the ability to \emph{decide} what to know (Stel 2016; cf. Beck 2008 on the conscious or unconscious inability-to-know). Thus, ignorance cannot be understood as pure absence of knowledge (Croissant 2014) or stupidity (Gupta 2012), but as something, which can be actively upheld and maintained or also manipulated (McGoey 2012b). By using ignorance as an active or passive strategy to cope, evade or engage with situations, individuals show varying degrees of agency remaining players ‘acting within relations of social inequality, asymmetry, and force’ (Ortner 2006, 139). Subjects are partially knowing (Giddens 1979), underlining not only the selective vision of ignorance, but also the individuals’ ability to act on and sometimes against the structures that made them (Ortner 2006, 110). Consequently, ignorance is entangled in human relations and interactions.
\par
While there is a difference between conscious and unconscious ignorance, which can lead to either active or passive strategies of ignorance (one can consciously not seek for information and knowledge which is available, thus rendering oneself passive; while one can unconsciously be ignorant due to a lack of knowledge and information available), this work rather focuses on the origins of ignorance. While trying to elaborate on how (un)conscious or active/passive certain ignorance is within the following analysis, the main argument in this work is the twofold nature of ignorance. First, it is created by a structural setup of the state and its agencies, which through their rules, frameworks and hierarchies create opportunities for ignorance to arise. This institutionalised ignorance (Beck 2008) is constructed, preserved and ---with time--- reproduces itself through the practiced (un)conscious and active or passive ignorance of street-level bureaucrats. Second,  personal values and opinions can create strategies of ignorance, which do not necessarily go against the structural setup, but have the potential to contest the state and policies. Thus, while state structures can create gaps of knowledge and ignorance  which the street-level bureaucrat may be unaware of, leading to a reproduction of structural violence through routine practices and the selected knowledge acquired by bureaucrats, bureaucrats also have the ability to consciously use ignorance to follow their own values, make work easier (see coping), or to reproduce the structural ignorance placed upon them in the first place. Similarly, migrants can make use of strategies of ignorance or being kept ignorant by the bureaucrat or the ‘state’.
\par
The key contribution of this work is that it highlights the violent outcomes of the interrelation between structural and individual ignorance, enhancing intangible practices. While ignorance embedded in public administration seems to be part of any given discretionary space, and thus still legally sound (though ultimately morally contestable), the created gap of information can force street-level bureaucrats to tinker practices which might follow the intention of policies, while at the same creating unexpected outcomes.
\par
Ignorance on the individual level can contest the structural side, to the (dis)advantage of the migrant with precarious legal status. Bureaucrats can resist legal guidelines and frameworks (consciously or not), including situations where bureaucrats ignore what should be known regarding their routines and legal procedures. This ‘stepping beyond their actual mandate’, thus moving in the realm of unsound practices, can lead to sanctions (job loss), but also highlights how ignorance can be morally charged and produce harm. Indeed, ignorance on an emotional level can block out and neutralise (Sykes and Matza 1957) everyday encounters with clients defined by ‘spontaneity, perishability, emotionality, vulnerability’ (Geertz 1973, 399). This production of ‘anonymisation of persons’ (ibid., 398) or distancing (Eule 2014) allows encounters not to become personal, while still processing cases as expected. Generally,  screening or shutting out are forms of denial where the individual only sees partially (Cohen 2001). 
\par
Finally, asking who doesn’t know and why not, can map the political geography ignorance creates. Bringing structural and individual aspects of ignorance together advances and understanding of how systems of oppression aim to silence the subject (Tuana 2008, 109) as the deliberate maintenance of un-knowledge and the withholding of information towards migrants incapacitates them.

\chapter{Methodological Framework}
This article is based on several months of ethnographic fieldwork in migration offices  (Switzerland, Latvia), border police units (Sweden) and border guard services (Latvia), as well as local police units (Switzerland). The selection of these three countries is based on the interest to study state agencies’ answer to irregular migration within the Schengen area, as well as to the given possibilities of accessing the field. While Sweden and Latvia are more centrally organised and Switzerland has a federal structure, further differences are found in the diverse geographical position (external and internal borders), organisational set-up, migration policies and migrant populations arriving. However, structural and individual strategies of ignorance play a crucial role in the everyday life of each group of bureaucrats, no matter how diverse their tasks and education are. Thus, this work can contribute by bringing forward crucial similarities, which have a strong effect not only on the bureaucrat, but also on case outcomes, thus finally on the migrant with precarious legal status.
\par
Between 2015 and 2017 data- deriving from participant observation, semi structured interviews and formal interviews or conversations- was collected. It was triangulated with the study of internal policy papers and case files (Flick 2011). The observed interactions between migrants and street-level bureaucrat, including (mobile) police officers or case workers in the office, have in common that the migrant subject was always in a precarious legal status. Either their asylum application was rejected and they were pushed to leave the country, they worked illegally (sometimes without knowing), or were placed in detention to await deportation. Regarding the used field notes, street-level bureaucrats have been named with capital letters and gender pronouns have been avoided.
\par
The collected recounted stories are situationally produced (Ewick and Silbey 1995) but embedded in a larger context disclosing power relations, which are hidden in social meaning. In order to study ignorance, participant observation helped to pinpoint moments where such ignorance became more evident. At the same time the interpretation of observed scenes and recounted stories connected to written statements and reports allows for a deeper understanding of ignorance already inherent in the government structure.

\chapter{Ignorance in Street-Level Encounters}
The following observation was collected during fieldwork at a Swiss Cantonal Police Station 2017. Depending on the size of the canton and the number of foreigners living in it, the cantonal police will have a specialised unit taking care of deportations and detentions, and also informing other police units about the current migration status of apprehended foreigners. The unit receives cases, and thus people, through the cantonal migration office. The excerpt highlights the manifold ways in which ignorance is present and how it is linked to indifference. 
    \blockquote{\emph{G} (a police officer of the migration police unit I am visiting) invites me to follow to the detention centre and quickly informs me about the detainee. Believing the detainee is from Eritrea, after briefly screening the files, \emph{G} translates a couple of sentences, via an online tool, in Tigrinya. I ask if \emph{G} knows about the detainee’s other language skills, but \emph{G} shrugs. Besides, the sentences translated only cover the section on health, “the rest will be fine […], we’ll see how it works out.” No phone translator is arranged. However, when we enter the meeting room, \emph{G} searches for a translation of the detention order in Amharic. After looking at the case file more closely \emph{G} realized that the detainee turns out to come from Ethiopia. When a translation cannot be found, \emph{G} does not bother and takes out an English one. I have time to screen the case file. The detainee will be sent to Germany. My attention wanders off to the great amount of available languages in which the detention order is available. All other forms are only available in German and English and are brought by the officer. \emph{G} remarks: “Well, if the intellect is missing, he can sit here for an hour, read and not understand anything. But with this translation he at least has something in his hands.“ \emph{G} also explains that the detainee should have received the deportation decision during his stay in the reception centre, and thus assumes he knows what is going to happen. When the detainee arrives, \emph{G} begins the conversation in German and after not hearing what the detainee answers switches to English. \emph{G}: „Do you speak English?“ - „Small.“ \emph{G}: „Small, ok. My name is \emph{G}. I am from the police. You know your situation?“ The detainee seems confused. \emph{G}: „No Asyl in Switzerland. Asyl is finished here.“ \emph{G} hands over the detention order. „This is my order. You sign? You go back Germany. You sign or not, what you want.“ After a couple of minutes of unsuccessful communication, the detainee, though explaining he does not understand what the form means, agrees to sign. \emph{G} has still not enquired about the detainee’s mother tongue and continues to believe it is Amharic. Therefore, \emph{G} hands him the next forms in English. Finally, the detainee asks if there is a Somali translation – he does not speak Amharic after all. \emph{G} looks at me and I nod.
\noindent The next form informs the detainee about the entry ban to Switzerland. \emph{G}: „The territory of Switzerland is closed for you. 3 years no Switzerland. Only information – migration gave it to you, just info. You sign or not?“ Again the detainee mentions he does not understand but signs. \emph{G} replies: „You understand? Yes, you understand. The territory of Switzerland is closed to you for 3 years and you can say to the problem what you want here (pointing to a line on the form). I explain you situation now. You understand.“ \emph{G} points to the line where the detainee could make a statement on the entry ban. „You can say sign or not sign.“ Again the detainee mentions he would sign even though he does not understand. Now, \emph{G} starts to get a bit insecure, decides to put ‘signature denied’ on the form and signs himself. Then \emph{G} looks at the detainee: „But now you know the situation in Switzerland.“ (Swiss Cantonal Police Unit 2017)}
\par
The fieldnote starts out with the indifferent attitude of \emph{G} towards the detainee. By not looking closely to the file, \emph{G} overlooks relevant information for the encounter and acts indifferent towards the case and the detainee. However, besides the disinterest towards the client, \emph{G} also does not deem the available information as relevant for the work task, thus ignoring it, keeping the file closed. The officer is confident to know enough to be prepared, which turns out to be wrong. Further, the quality of the encounter is of no interest to \emph{G}, reducing the level of information exchange to a minimum, as \emph{G} assumes all relevant information has been given to the detainee beforehand. Thus, what starts out as an indifferent attitude, discloses several strategies of structural and individual ignorance going beyond what indifference is able to explain.
\par
While the active refusal to understand can be seen as an act of agency, moments where migrants genuinely do not understand are not. The detainee is not aware of what will happen to him and voices his struggle to understand. However, he is willing to sign the forms, disregarding his lack of knowledge. He is literally depending on the knowledge of the officer, who decides how much to share, since he is detained, without access to other knowledge. As I encounter many of these interactions, it is valid to mention that power inequalities (Galtung 1969) and thus structural violence are very much present at any moment, as each interaction is characterised by different amounts of information handed out, thus decapacitating the clients to various degrees. During hearings of the Swedish Border Police regarding the prolongation of detention the officers clearly explain that no further questions will be discussed. Any attempt to break this rule is met with firm refusal to answer and repetition of this rule (Fieldnotes, 2017). It is they who decide how much information is shared and it is them who decide if the other has understood.
\par
While the bureaucrat is able to withhold information at any time, leaving the client in a state of un-knowledge or partial knowledge and thus uncertainty, the migrant has only limited influence. This is partly supported by the procedures of the system, in which the officer is the last one in a line of bureaucrats who processes the case, thus accepting it without much reflexivity. (S)he does what is expected of her/him.
\par
Street-level bureaucrats often explain their blocking of client’s questions with their lack of responsibility. To them more knowledge given to the migrant would not make a difference (see \emph{G}) as they perceive cases as closed and clients should understand that ‘this’ is the end of all procedure, that it is time to leave. B, a caseworker in a Swiss cantonal migration office, explains: '\textit{Other colleagues might read through the asylum application interview, but I do not. It is of no interest to me, it is all lies anyway (laughs). I just know, this person has to go and I do it. The national migration office can take care of the rest}' (Fieldnotes, 2016). Like many other colleagues the caseworker simply practices the 'won’t tell, [\textellipsis] don’t know, and frankly [\textellipsis] don’t care'--attitude (Bauman 2008, 70). Knowledge of the case is irrelevant, as according to B one does not need to know a case in order to process it. 
\par
Also, all interlocutors imply that acquiring of more information might not reduce uncertainty or ignorance, but can lead to confusion where information conflicts (Smithson 2010). Ignorance is thus presented as a strategy to avoid complicated encounters to supposedly ‘help’ the client to understand. The decisions taken are within the given framework, and thus resemble the everyday discretionary choices bureaucrats make.
\par
Refusing to get acquainted with a case more than the officers deem to be necessary is thus a ‘professional’ decision developed with experience. Many times, I get a quick shrug when asking about details of the cases, added by a short: \enquote{I do not know} or \enquote{I do not care}.
\par
Reducing the intake of information might facilitate the workload as it takes less time to get familiar with a case, thus only ‘relevant information’ to fulfil a task is screened. Officers working with detention and deportation do not need to know the entire asylum request, the stories told and the reasons for rejection. In their everyday work, they are the ones \enquote{executing orders} (Fieldnotes, 2016--2017). Taking in more information than is relevant to implement their work is time consuming. While not necessarily misguiding the clients of public administration, the ignorance of personal stories and information – for whatever reason - adds a moral value to the denial of migrants’ agency. Like \textit{F} in the first fieldnote, limiting the encounter to a set of simple questions asked to the detainee, keeps the conversation and discussion to a minimum. Uncomfortable knowledge is kept at bay or is dismissed (Rayner 2012).
\par
However, while police officers/border guards are not bound to double-check cases and screen decisions already taken by the migration office or migration courts, their work needs to be grounded on correct decisions. Some information is relevant to perform well, and thus the process of deciding what to read and what not is crucial. Also, by reading decisions and files, officers could have the opportunity to function as a last control mechanism, while also being emotionally and professionally responsive to their opposites.
\par
Besides the active ignorance of available facts, the fieldnote elucidates \emph{G}’s lack of knowledge, increasing the probability of flaws. While \emph{G} does not seem to bother to get a decent translation, \emph{G} is also unaware of the existing languages of the detention orders, thus acting on partial knowledge or even non-knowledge. This negligence can cause serious trouble for the processing of cases and of course for the individual who might be detained or deported. While individuals cannot and do not know everything (Douglas 1986; Croissant 2014), \emph{G}’s attitude goes beyond simply being careless, yet remaining entirely confident, assuring me the meeting went as expected. This reduction of the migrant  to a passive element denies them the same capacities and reproduces structural violence (ibid., Gupta 2012). Despite getting familiar with someone’s case in order to address them correctly as sign of respect (cf. Smithson 1989), \emph{G} also denies the detainee a proper translation, creating unsound administrative practices caused by ignorance. Legally, \emph{G} has to meet the client in detention, who has a right to be heard, though due to the ignorant strategies it actually loses its validity, as the client does not understand what will happen to him.
\par
Street-level encounters strongly reflect the function of ignorance as reinforcement of traditional values and maintenance of privileged positions and expertise (Moore and Tumin 1949). Obviously, there is different access to knowledge and it is the street-level bureaucrat who can choose to disclose information in order to fulfil their task or follow their own moral sentiments (or not). At the same time, it underlines the broad discretionary space they have in defining their tasks. As such, informing the detainee is highly subjective. To \emph{G} the tasks are fulfilled sufficiently. Where street-level bureaucrats follow readymade patterns and engage in a common idea of how the job is done, ‘the individual’s notions of right and wrong are rigidified [and] susceptibility to new knowledge and influence is minimized’ (ibid., 791).
\par
Also, many bureaucrats assume that all of their clients lie (see \emph{B}) or are well-informed (\emph{G}), accounting for the unwillingness of street-level bureaucrats to repeat explanations on procedures and thus maintaining a state of ignorance among detainees. Some officers preserve stereotypes, depending on narrowly defined roles, reducing information on the otherwise often personal encounters. This type of ignorance is required, ‘whenever knowledge would impair impersonal fulfilment or duties’ (Moore and Tumin 1949, 793). For officers, ignorance often functions as a positive and active element of operating structures, thus does not leave the structure dysfunctional (ibid., 795). In contrast, the migrant experiences a great decrease in agency. \emph{G}’s rhetoric question ‘You understand?’ and own answer ‘Yes you understand’ are just one example of many encounters, where bureaucrats did not listen sufficiently. The asymmetry of power structures visible in these encounter prove how quickly ignorant behaviour is produced and used in everyday encounters. For \emph{G} the meeting is one of many, a practiced routine, thus so banal that the actual execution of meetings easily whitewashes structural violence (Proctor 2008; Slater 2012).
\par
In other situations, officers might resort to more passive ignorance, letting clients talk and ask questions, without taking the stories and concerns into account. F resumes work, filling out papers, while the client continues talking and explaining. Here the client’s voice might be allowed, but not heard. It is not simply indifference but the assumption of irrelevance that lead officers to neglect knowledge and information. It is also a strategy to meet the expected outcome: filing forms, thus keeping up productive tasks, is combined with teaching the client a lesson: no matter how much one complains, what story is told, there is nothing that can be done. Ignorance becomes a productive asset to justify and evade responsibility (McGoey 2012a; Stel 2016). The stalling (Stel 2016) and stonewalling (Sedgwick 1990 see also F) is an intentional strategy to reduce the intake of information, thus highlighting the resistance to get involved too deeply, which functions as a coping mechanism (de Graaf, Huberts, and Smulders 2014; Blau and Meyer 1987). 
    \blockquote{\emph{O}: ‘The back side of the job is: If you see more than the usual human being sees, your mind set changes. One is more involved and knows more.’ (Field Notes, Swedish Border Police 2017)}
\noindent What \emph{O} refers to is the struggle to leave work with a free and unbothered mind. \emph{O} previously worked as border control staff at the Airport, where \enquote{one hands over the case to another person. It is easier to switch off and the next day one comes back and one has something new.} Instead, in the current job as a regular border police case worker \emph{O} follows cases until deportation, which \enquote{is sometimes not that easy} (Fieldnotes, 2016). When thinking about the caseload, \emph{O} mentions having been involved in about one hundred cases since starting the job 5 months ago. ‘I should not think about it. It is nearly the same as to think about the universe.’ However, it is not only the sheer amount of information, but also the personal involvement in cases, which makes it difficult for street-level bureaucrats to ‘switch off’. Efforts to blind out personal stories and values, which might interfere with their work is met with strategies to reduce involvement.
\par
While the ‘shutting out’ of daily work experiences is connected to taking a break from the ‘job’, tasks and eventually unpleasant encounters, the blocking out of personal views reduces friction regarding the execution of tasks. However, being ignorant towards ones own emotional and political viewpoints might reduce the ability of reflection. Declining to reflect on ones own positions during work might disrupt the carefully maintained work free zone of private life, underlining the struggle officers might go through to be able to ignore.
\par
However, the use of ignorance can also be directed against the agency, the structure and thus the state, highlighting individual ignorance regarding bureaucrats’ own views and norms. Refusing to take up orders and going against guidelines and regulations is an active decision to ignore, and to follow own hidden transcripts (Scott 1990). This bears the danger of taking up more discretion than the structures grant, and results in less common, but more disruptive moments.
    \blockquote{A person from the National Swiss Migration Agency calls the cantonal migration office – a man with an Italian residence permit was apprehended, but the cantonal office decided not to take any actions because he has refugee status and an Italian residence permit. The national office wonders why they did not detain, as he could still be returned to Italy. U  later tells me: “They just wanted to get rid of him, or put him in jail, but he had documents. If they seem valid, one has to let him go. The Italians should have told the national office about his documents.” U walks over to \emph{W}’s office and summarises the call. \emph{W}: “Everything is perfectly fine. It is not in our competence and does not interest us. And the National Office can surely tell us what they think we should do, but we will do what we want to. The use of coercive measures lies in the competence of the canton. He is recognised as a refugee. We could have detained him, but why would we? For us he is a tourist and it would not make any sense to detain him. And we are not talking about a package or something, but about a human being. Also, he could sue us in the end and then we would eventually have to pay him a compensation for the detention.” (Field notes, Swiss Cantonal Migration Office 2016)}
\par
The excerpt elucidates how ignorance can have a positive outcome for the otherwise often marginalised client. While detention as measure can be used, the decision lies in the hands of the cantonal migration office. Thus, it highly depends on personal decisions and the use of discretionary spaces (Lipsky 2010; Eule 2014). Here, the bureaucrat voices two reasons for refuting the national office’s suggestion. It is a will to acknowledge the impact detention has on a human being, but also the assumed costs if they fail to proof the necessity for detention in front of the court. Ignoring the general routine, guided by the national office, \textit{W} actively goes against their way of handling it, thus ignoring generally accepted practices to the advantage of the client and finally their own office. A second encounter between \emph{W} and another Dublin case elucidates what Smithson (2008) calls ‘arrangement of ignorance’. The client has been deported from Sweden to Switzerland, as the latter is responsible for the decision on the case. However, the client was already rejected. Now, by turning up again, the officer could detain him. However, \textit{W} openly explains, as if talking to himself, what options could follow: \enquote{You will be detained and sent back to your home country, if you turn up again. But if you would abscond, there is nothing we can do…} (Field notes Swiss Cantonal Migration Office 2016). In contrast to situations where officers keep information hidden (e.g. not telling about deportation dates), \emph{W} openly shares what will happen, going against the actual rules, which \emph{W} certainty is aware of. \emph{W} discloses information, which should not be given, ignoring the fact that it would be counterproductive to his actual task: to implement deportation orders. Instead, \emph{W} openly shares knowledge and information, as if the client was not visible (Smithson 2008), using ignorance as strategic ploy (Proctor 2008). Ignorance can thus also become knowledge (McGoey 2012a).
\par
In yet another case, two Swedish border police officers admit to have shuffled cases under their piles of documents, in order to ‘forget about them’, to either give migrants more time before a deportation or even to make Dublin deportation cases a national responsibility (if timeframes are not respected). Going against legal practice because of practical thinking or bureaucrats’ own ideas of right and wrong brings forward an individual set of thoughts and a morally charged work environment, where structural violence is strongly intertwined with strategies of ignorance. Ignorance, acted out passively or actively, consciously or not, always ends up in a highly uncertain outcome for the migrant. Looking at everyday encounters of bureaucrats and their ‘clients’ enabled me to define how far ignorance is used and produced and for which reasons. Surely ignorance is used as strategy to refrain from accountability and responsibility (cf. Sykes and Matza 1957), avoiding emotional responses and moral assessment (Smithson, 2008). What might be used as a strategy to avoid internal conflicts in organisations (Smithson, 2008), such as managing heavy workload, ends up being a key demarcation for the development structural violence.

\section{Migrants' Use of Ignorance}
Regularly I observed interactions between migrants and bureaucrats involving questions regarding their legal advisors. Their dependence on third parties who are supposed to help them appeal the case often results in experiences of financial exploitation and partial knowledge. Again, the acting upon partial knowledge, clearly underlining unequal positions and thus structural violence, impacts on the clients’ ability to claim agency and causes great uncertainty even when following legal bureaucratic avenues.
\par
Though focus is put on the institutional and bureaucrats’ use and production of ignorance and the harmful outcome for migrants, it is relevant to shortly contextualise migrants’ strategies of ignorance. These often resemble street-level bureaucrat’s practices and the interplay of bureaucrats’ and migrants’ ignorant strategies generates unintended outcomes (Smithson 2008). Migrants too leave out relevant information (about their origin, age, journeys), ignore the information they receive and act upon what they deem best for themselves. Stel (2016) uncovers agnotological responses of Lebanon’s Palestinian refugees living in unofficial camps. Using the institutional ambiguity of the camps’ existence, migrants’ strategies often are closely connected to resisting the structural violence and uncertainty they face in their everyday lives. Their strategies are more clearly directed against the state apparatus, regardless of their active or passive nature. Often, their use of ignorance is a reaction to the uncertainty created through bureaucratic uses of ignorance.
\par
However, the very different outcome of ignorant behaviour for the migrant reflects the power structures that are at play here. Not only does the client depend on the willingness of the bureaucrat to inform them, but also acting upon partial or un-knowledge can lead to severe constraints, displaying their precarious situation. Should one abscond because a caseworker suggests it (see \emph{W})? Is it helpful not to disclose the identity and ‘refuse’ cooperation in order to hinder deportation, while increasing the risk of being detained? Even though a certain amount of agency is kept by every individual (Ortner 2006), twisting the power-play to their favour often comes with a high price of remaining in precarious legal status. (Un)conscious ignorance is reflected in Swedish border police and migrant encounters, where many migrants are apprehended at work, because legally they are not allowed to work according to the migration office. Assuming that the personal card handed out by the tax office and paying tax allowed them to find a job, many migrants are detained for their breach of law. Latvian border guards mention the unwillingness of Vietnamese detainees to share information; they do not tell their names and do not contact the Vietnamese embassy for paperwork in order to leave. Other tactics of ignoring deportation orders are handing out wrong addresses, being on the move or absconding. Some bureaucrats also mention women getting pregnant, absconding until they are too advanced in their pregnancy to be deported.

\section{Ignorance as Inherit Feature of the State}
Street-level bureaucrats excuse their work by refusing knowledge, thus responsibility, therefore showing indifference, but also reveal their unintentional lack of knowledge. Structurally created ignorance allows bureaucrats to complete their tasks, while blinding out ‘unnecessary information, deemed irrelevant to the job.  It creates ‘conditions which ensure its continuance’ (Frye 1983). Consequently, ignorance adds a wilful side (Beck 2008), which is systematically maintained (Smithson 2008), thus adding an active component to the creation of structural violence experienced by migrants.
\par
While inheriting traditional values of the organisations assures the system’s continuance and hides the punitive character of the state (Slater 2012), street-level bureaucrats are trained by them, learning to put trust into a system which equips them with knowledge on everyday tasks. Hence, bureaucrats automatically make use of structures of ignorance inherent to the state, embedding them in their work. By engaging in their routines and not asking questions, bureaucrats maintain the state of not-knowing. They do not simply create their own spaces of ignorance, which they willingly foster to either face or keep out of moral dilemmas, uneasy cases and thoughts. Instead, the structure of a state agency impacts on their ignorance, shapes it and eventually maintains it in a similar way as bureaucrats manipulate and control migrants’ knowledge. Thus, ignorance is co-produced by policies, laws, migrants as well as bureaucrats, but deeply embedded on a structural level.
\par
Generally, the most common tasks and practices will be solved through ‘learning-by-doing’, rather than through prior study. This leads to certain practices being continued, while others are not. This learning process is not monitored, and by grounding the major work processes on an experience-only and learning-by-doing structure, training programs and cooperation networks seem to deliberately accept and even institutionalise ignorance (Slater 2012).
\par
While a Latvian border guard explained that before the restructuring each officer was responsible for a case from the very beginning to the end (meaning deportation), the division of labour creates spaces of ignorance within the agency. Officers keep working in their sometimes very narrow environment, encouraging a narrow mind-set. The dispersal of responsibilities surely supports such behaviour, as does the division of work processes.
\par
On a structural level, ignorance might be not only accepted, but even actively encouraged, because knowledge is associated with power and thus can become a danger (Proctor 2008). Therefore, keeping staff in doubt about possible practices might be a means to not only disperse accountability but also avoid conflicting guidelines or practices. Certain information is withheld by superiors and not shared with the street-level bureaucrat. Hence, the manipulation of knowledge of others can be observed along a line, from structural to individual level.
\par
While some officers are very keen to avoid getting too involved in a case, the system’s general structure can make it very difficult for more ‘interested’ employees to get to know a case. A Latvian risk analysis specialist underlines how he had to study by himself. No support or training were available after a certain level, thus forcing him to find new sources of knowledge (Field notes 2016). Also, for the Swiss and Swedish bureaucrats it is very important to have as much information on a person before they apprehend, detain and eventually deport them. Detailed information, e.g. about the health, potential aggression or physical abilities, is not only relevant to maintain their own safety, but also to guarantee the migrant’s well-being. However, knowledge in migration office differs from that available in border police units, due to different databases and information access. Slow bureaucratic chains of communication (see Borrelli, 2018) further contribute to bureaucrats' acceptance that not-knowing needs to be accepted in certain moments.
\par
Also, the bureaucratic structure might support ignorant behaviour on side of the migrant. In 2016 the National Swiss Migration Office decided to financially punish cantons which have not been able to process Dublin returns in their given time frame. The offices have six months to process and send back the client in order to get reimbursed for the costs on national level. If they fail the costs remain a responsibility of the canton. However, if a person absconds before the six months are over, the time for a return will be extended to a total of 18 months. Resorting to financial punishment on national level encourages cantonal offices to, indirectly ‘support’ absconding (see \emph{W}). While the organisation of the agency manipulates knowledge and information received by the bureaucrat (or not), the individual and personal interaction with the client is characterised by a consecutive manipulation (Proctor 2008, 24). Handing out partial knowledge on possible detention might tip off the migrant enough to decide to abscond. Where a lack of knowledge can actually help the bureaucrat to differentiate between what is important and not in order to keep the system running, the migrant often wishes to receive as much information as possible. Hence, social practices of ignorance bear the imprint of power relations and reproduce taken-for granted worlds (Ewick and Silbey 1995, 215; Smithson 2008, 218 f). The cultivation of ignorance helps the state agency to excuse their employees that they did not know better (Proctor 2008; McGoey 2012a, 2012b; Michaels 2008) and functions as social control.
\par
‘Ignorance is frequently constructed and actively preserved, and is linked to issues of cognitive authority, doubt trust, silencing, and uncertainty […] [thus] intersects with systems of oppression’ (Tuana 2008, 109). Power relations are embedded in an institutional order (Giddens 1979) and play out in the actual social interactions on the ground. Ignorance can be seen as a means of power relations, even if not used consciously. It is often deeply embedded in the structures of the agency and thus shapes the individual’s disposition (Ortner 2006) and traditions. These individuals are finally guided by the embedded ignorance and often end up accepting it. However, the individual always maintains a certain range of agency (Giddens 1979; Scott 1990), eventually following their own ‘hidden transcripts’ (Scott 1990), made possible through the structures allowing for discretion and thus opening up the potential use of ignorance against the state.

\chapter{Conclusion --- What Is The Cost of Ignorance?}
This article has attempted to map how structural and individual strategies of ignorance cause state practices to become highly intangible and unreadable. It not only places ignorance as constitutive strategy of the state but highlights that ignorance and being ignorant is used as legitimate strategy in avoiding responsibility towards migrants. The bureaucrats discussed here have the particular task to detect, detain and deport migrants with precarious legal status. At times, they lack the professional knowledge to fully act, but are still expected to and at the same time ever-changing policies make it difficult for them to do so (Fieldnotes, Sweden 2017). While bureaucratic procedures are generally acknowledged to change at a high pace (cf. Eule, et al., 2018), the field of migration is characterised by an increasing restrictive position-both in the researched states and many other Schengen Member States, as well as  increasing politicisation.
\par
As a concept, ignorance explains how knowledge is manipulated and how non-knowledge is produced, used, reproduced and acted upon by state agents, as well as migrants. While ignorance is an integral part of the state, as practices are based on knowledge and the lack thereof, street-level bureaucrats and migrants can partly use the structurally embedded strategies of ignorance to regain agency. While bureaucrats use ignorance to manage their tasks, they also engage in such strategies to reduce the emotional labour, as well as to follow their own ideals and values, or resisting against what they deem unfair state practices. Presented data shows that bureaucratic agencies dealing with the active implementation of detention and deportation orders, which both have physical consequences, often underline their role as just ‘doing the job, implementing orders’. Through this distancing between the ones responsible taking the orders and themselves, bureaucrats deny migrants the possibility to act. Thus, their behaviour shapes migrants’ behaviour, but also silences them. Migrants are being kept ignorant and they might base their decisions on the lack of knowledge and the manipulated information they receive. This in turn strongly impacts on their uncertain future, as people act upon knowledge but also on the un-knowledge they possess.
\par
Uses of ignorance manifest at times in the pure neglect of actual procedures and practices, thus highlighting the maliciousness of the bureaucratic encounter. Showing how structural and individual strategies of ignorance play out and are intertwined, highlights how structural violence is not only already embedded in the agencies’ structures, but also how it is reproduced and its effects multiplied. Relating individual with structural aspects of ignorance in bureaucratic everyday work underlines how a banal, but severe reproduction of harmful effects comes into being. Both sources of ignorance affect each other and thus can be influenced and manipulated.
\par
This work has tried to show that a gap between knowing and un-knowing does not simply come into being through individual decisions only. It rather manifests through the multiple ways un-knowledge is produced, maintained and reproduced. Even active striving for a reduction of un-knowledge on both sides, the migrant and the bureaucrat, might not reduce their state of deprivation. Instead, the state can be understood as ‘the ignorant’, producing and facilitating moments of ignorance, though not fully capable of entirely controlling its use and outcomes. Ignorance is thus a constitutive part of the system. At the same time, the concept of ignorance, in contrast to indifference, brings back responsibility to the individual using it. It does not deny agency, but allows for a distinction of uses of ignorance, thus demarcating when ignorance has been used in what way to distinguish between acts of resistance and acts of neglect.