    \begin{adjustwidth}{20mm}{20mm}
\label{paper4:abstract}
    \bigskip
    \begin{otherlanguage}{english}
    {\small
%    \fadebreak
%%%%%%%%%%%%%%%%%%%%%%%%%%%%%%%%%%%%%%%%%%%%%%%%%%%%%%%%
%%%%%%%%%%%%%%%% START ABSTRACT.tex %%%%%%%%%%%%%%%%%%%%
%%%%%%%%%%%%%%%%%%%%%%%%%%%%%%%%%%%%%%%%%%%%%%%%%%%%%%%%

\noindent Street-level bureaucrats working in the field of migration enforcement have the uneasy task of finding irregularised migrants and processing their cases – often until deportation. As the encounters are unforeseeable and characterised by tension and emotions, bureaucrats develop practices and strategies, which help them to manage the often very personal encounters. Besides the frequently debated strategies summarised under the term ‘copying mechanisms’ and the problem of ‘dirty’ or many hands, ignorance as a tactic in the daily work of bureaucrats has not been studied to a sufficient extent.
\par
This work looks at how ignorance, including  deliberate not-knowing or blinding out, as well as undeliberate partial-knowing or being kept ignorant, is used in public administration, through multi-sited, ethnographic fieldwork in migration offices and border police/guard offices of three Schengen Member States: Sweden, Switzerland and Latvia. It distinguishes between structural and individual ignorance, which both have the ability to limit migrant’s agency. Further, by analysing their intertwined relation, this article furthers our understanding of how uncertainty and a lack of accountability become results of everyday bureaucratic encounters. Ignorance thus  obscures state practices, subjecting migrants with precarious legal status  to structural violence.

\smallskip
\noindent\rule{\linewidth}{1pt}

%%%%% KEYWORDS
%
\noindent{\bfseries Keywords:}\hspace*{0.75em}{% list keywords below
bureaucracy;
agnotology;
ignorance;
migration;
public administration;
discretion}.





    } % ends font family

%    \fadebreak

    \end{otherlanguage}

    \end{adjustwidth}