\label{paper2:references}
\begin{thebibliography}{99}
%%%%%%%%%%%%%%%%%%%%%%%%%%%%%%%%%%%%%%%%%%%%%%%%%%%%%%%%
%%%%%%%%%%%%%%% START 9_references.tex %%%%%%%%%%%%%%%%%
%%%%%%%%%%%%%%%%%%%%%%%%%%%%%%%%%%%%%%%%%%%%%%%%%%%%%%%%

\item Adams, R. (1996). \textit{Social Work and Empowerment}. New York: MacMillan Press.  
\item Andersen, D. (2015). Stories of change in drug treatment: A narrative analysis of 'whats' and 'hows' in institutional storytelling. \textit{Sociology of Health and Illness}, 37(5), 668-82.
\item Bacchi, C. (2009). \textit{Analysing Policy: What’s the problem represented to be?} Frenchs Forest: Australia. 
\item Bjerge, B. \& Bjerregaard, T. (2017). The twilight zone: Paradoxes of practicing reform. \textit{Journal of Organizational Ethnography}, 6(2): 100-115. 
\item Bjerge, B. \& Rowe, M. (2017). Inside the black box of public service change. \textit{Journal of Organizational Ethnography}, 6(2), 62-67.
\item Bloche, M. G. \& Cournos, F (1990). Mental health policy for the 1990s: Tinkering in the interstices. \textit{Journal of Health Politics Policy Law}, 15, 387–411.
\item Brodkin, E. (2006). Bureaucracy redux: Management reformism and the welfare state. \textit{Journal of Public Administration Research and Theory}, 17(1), 1-17.
\item Christensen, L. (2017). Metaphors of change. Descriptions of changes within the practice of social work for socially marginalized people. \textit{Journal of Organizational Ethnography}, 6(2), 62-67.
\item Durose, C. (2011). Revisiting Lipsky: Front-line work in UK local governance. \textit{Political Studies}, 59, 978-955.
\item Eysenck, H. J. (1976). \textit{Case studies in behaviour therapy}. London: Routledge. 
\item Flyvbjerg, B. (2006). Five misunderstandings about case-study research. \textit{Qualitative inquiry}, 12(2), 219-245.
\item Frank, V. A. \& Bjerge, B. (2011). Empowerment in drug treatment: Dilemmas in implementing policy in welfare institutions. \textit{Social Science \& Medicine}, 73(2), 201-208.
\item Grommé, F. (2015). Turning aggression into an object of intervention: Tinkering in a crime control pilot study. \textit{Science as Culture}, 24(2), 227-247.
\item Gubrium, J. F. \& Holstein, J. A. (2001). \textit{Institutional selves: Troubled identities in a postmodern world}. New York, NY: Oxford University Press.
\item Jenkins, R. (2000). Categorization: Identity, social process and epistemology. \textit{Current Sociology}, 48(3), 7-25.
\item Knorr, K. D. (1979). Tinkering towards success: Prelude to a theory of scientific practice. \textit{Theory and Society}, 8(3), 347–376.
\item Latour, B. \& Woolgar, S. (1986). \textit{Laboratory life: The construction of scientific facts}. Princeton: Princeton University Press. 
\item Lipsky, M. (1980).\textit{ Street-level bureaucracy. Dilemmas of the individual in public service}. New York, NY: Russel Sage Foundation.
\item Lydahl, D. (2017). Same \textit{and} different. Perceptions of the introduction of person-centred care as standard healthcare. \textit{Gothenburg Studies in Sociology}, Ph.D. thesis, University of Gothenburg.
\item Mol, A., Moser, I., \& Pols, J. (Eds.). (2010). \textit{Care in practice: On tinkering in clinics, homes and farms}. Bielefeld: transcript Verlag.
\item Møller, M. Ø, \& Harrits, G. S. (2013). Constructing at-risk target groups. \textit{Critical Policy Studies}, 7(2), 155-176.
\item Nicolini, D. (2009). Zooming in and out: Studying practices by switching theoretical lenses and
trailing connections. \textit{Organizational Studies}, 30(12), 1391-1418.
\item Nielsen, B. \& Houborg, E. (2015). Addiction, drugs and experimentation: Methadone maintenance treatment between 'in here' \& 'out there'. \textit{Contemporary Drug Problems}, 42(4); 274-288.
\item Oute, J. (2018). ‘It’s a bit like being a parent’: A discourse analysis of how nursing identify can contextualize patient involvement in Danish psychiatry. \textit{Nordic Journal of Nursing Research}, 38(1), 1-10. 
\item Oute, J. \& B. Bjerge. (2017) What role does employment play in dual recovery? A qualitative meta-synthesis of cross-cutting studies treating substance use treatment, psychiatry and unemployment services. \textit{Advances in Dual Diagnosis}, 10, 3: 105-119.
\item Oute, J. \& B. Bjerge. (Forthcoming). Ethnographic reflections on accessibility to care services. Journal of Organizational Ethnography. 
\item Payne, M. (2014) [1990]. \textit{Modern social work theory}. Basingstoke: Palgrave Macmillan.
\item Resnick, M. \& Rosenbaum, E. (2013). Designing for tinkerability. Pp. 163-181 in: Honey, M. \& Kanter, D. E. (Eds.), \textit{Design, make, play: Growing the next generation of STEM innovators}. New York, NY: Routledge. 
\item Rossen, C. B. (2016). \textit{Pakkeudredning i det danske sundhedsvæsen: en analyse af udredningsprocesser i pakkeforløb}. PhD thesis. Odense: Institut for Regional Sundhedsforskning, Syddansk Universitet.
\item Spector, M. \& J. I. Kitsuse (1977). \textit{Constructing Social Problems}. San Francisco, CA: Benjamin-Cummings Publishing Company.
\item Stax, T.B. (2003). Fordi ingen er ens – eller...? En analyse af tre hjemløse klienters strategier på et lokalcenter. Pp. 164-191 in: Järvinen, M. \& N. Mik-Meyer, N. (Eds.) \textit{At skabe en klient}. København: Hans Reitzels Forlag.
\item Vohnsen, Nina Holm (2015) Street-level Planning: The shifty nature of “local knowledge and practice”. \textit{Journal of Organizational Ethnography}, 4(2): 147-161.
\item Vohnsen, N. H. (2017). The Absurdity of Bureaucracy. How implementation works. Manchester, UK: \textit{Manchester University Press}.
\item Winance, M. (2010). Care and disability. Practices of experimenting, tinkering with, and arranging people and technical aids. Pp. 93-117 in: Mol, A, Moser, J and Pols, J (Eds.). \textit{Care in Practice. On tinkering in clinics, homes and farms}. Bielefeld: transcript Verlag.

%##################################################################################################################################################################################################################################
\end{thebibliography}