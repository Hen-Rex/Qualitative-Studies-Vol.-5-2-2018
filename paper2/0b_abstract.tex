    \begin{adjustwidth}{20mm}{20mm}
\label{paper2:abstract}
    \bigskip
    \begin{otherlanguage}{english}
    {\small
%    \fadebreak
%%%%%%%%%%%%%%%%%%%%%%%%%%%%%%%%%%%%%%%%%%%%%%%%%%%%%%%%
%%%%%%%%%%%%%%%% START ABSTRACT.tex %%%%%%%%%%%%%%%%%%%%
%%%%%%%%%%%%%%%%%%%%%%%%%%%%%%%%%%%%%%%%%%%%%%%%%%%%%%%%

\noindent Citizens with complex problems are often in touch with different welfare services and administrative systems in order to receive the help, they need. Sometimes these services overlap and sometimes they conflict. The lack of ready-made services to match the complex, multiple, and often shifting needs of citizens with complex problems presents a challenge to caseworkers in the welfare system. In this article, we zoom in on the management of a single user´s case, in order to examine in detail how caseworkers nevertheless make casework ‘work’. We employ the concept of ‘\textit{tinkering}’ to highlight the ad hoc and experimental way in which caseworkers work towards adjusting services to the unique case of such citizens. Tinkering has previously been used in studies of human-technology relations, among others in studies of care-work in the welfare system. In this paper, we employ the concept to capture and describe a style of working that, although not a formally recognized method, might be recognizable to many caseworkers in the welfare system. We show how tinkering involves the negotiation of three topics of concern, namely the availability of services, the potentials of services to be adjusted to the particular problems of the citizen, and finally, the potential for interpreting these problems and the citizen’s needs in a way that they match the service. We further demonstrate that casework tinkering involves both short-term and long-term negotiation of services. Firstly, tinkering is involved in the continual adjustment and tailoring of services to the immediate needs of the citizen, but secondly, it also speaks to a more proactive process of working towards a more long-term goal. 

\smallskip
\noindent\rule{\linewidth}{1pt}

%%%%% KEYWORDS
%
\noindent{\bfseries Keywords:}\hspace*{0.75em}{% list keywords below
citizens with complex problems;
casework;
case studies;
tinkering}.



    } % ends font family

%    \fadebreak

    \end{otherlanguage}

    \end{adjustwidth}