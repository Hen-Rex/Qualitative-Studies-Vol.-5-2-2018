\subsection{About the Authors}
%    \marginnote{please write a mini-biography/description of yourself, qualifications, and current work, and anything else you might think of :)}
\label{paper2:colophon}
%%%%%%%%%%%%%%%%%%%%%%%%%%%%%%%%%%%%%%%%%%%%%%%%%%%%%%%%
%%%%%%%%%%%%% START ACKNOWLEDGEMENTS.tex %%%%%%%%%%%%%%%
%%%%%%%%%%%%%%%%%%%%%%%%%%%%%%%%%%%%%%%%%%%%%%%%%%%%%%%%

\textbf{Maj Nygaard-Christensen}, PhD, is Assistant Professor at \textit{Centre for Alcohol and Drug Research}, Aarhus University. She holds an MA and a PhD in anthropology. Her research interests are ethnographically oriented studies  of political and development interventions and policy implementation processes. Her current research projects focus on marginalized Greenlanders in Denmark, poverty, and violence among socially marginalized citizens.
\\
\textbf{Bagga Bjerge}, PhD, is Associate Professor at \textit{Centre for Alcohol and Drug Research}, Aarhus University. She is trained within anthropology and sociology. Her research activities are mainly based on qualitative methods focusing on policies and policy implementation, bureaucracy, social work as well as social marginalization. Currently, her research projects focus on poverty, drug scenes in Copenhagen, marginalized Greenlanders in Denmark and the management of citizens with complex cases.
\\
\textbf{Jeppe Oute}, PhD, is Assistant Professor at \textit{Centre for Alcohol and Drug Research, Aarhus University}. He holds an MA in Educational Anthropology and a PhD in Social studies in Medicine. Oute’s research interests are twofold. His research focuses on the social effects of mental health policy in psychiatric and drug treatment practices and on the societal and professional-institutional contingency of recovery and stigma in the mental health field.