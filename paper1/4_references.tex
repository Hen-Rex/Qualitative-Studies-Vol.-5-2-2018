\label{paper1:references}
\begin{thebibliography}{99}
%%%%%%%%%%%%%%%%%%%%%%%%%%%%%%%%%%%%%%%%%%%%%%%%%%%%%%%%
%%%%%%%%%%%%%%% START 9_references.tex %%%%%%%%%%%%%%%%%
%%%%%%%%%%%%%%%%%%%%%%%%%%%%%%%%%%%%%%%%%%%%%%%%%%%%%%%%

\item Bartfeld, J. (2003). Falling through the cracks: Gaps in child support among welfare recipients. \textit{Journal of Marriage and Family} 65 (1), 72-89.
\item Bjerge, B. \& Bjerregaard, T. (2017). The twilight of reform: Organisational paradoxes as practice, \textit{Journal of Organizational Ethnography}. 6 ( 2), 100-115.
\item Bjerge, B. \& Rowe, M. (2017). The twilight zone: paradoxes of practicing reform, \textit{Journal of Organizational Ethnography}. 6 (2), 62-67.
\item Bjerge, B. \& Nielsen, B. (2014). Empowered and self-managing users in methodone treatment? \textit{European Journal of Social Work}, 17 (1) 74-87
\item Brown, L. D. (1983). \textit{Managing conflict at organizational interfaces}. Addison Wesley Publishing Company.
\item Burns, J. K. (2009). Mental health and inequity: A human rights approach to inequality, discrimination, and mental disability. \textit{Health and Human Rights} 11 (2), 19-31.
\item Certeau, M. D. (1984). \textit{The practice of everyday life}. Berkeley.
\item Cristofoli, D., M. Meneguzzo, and N. Riccucci. 2017. “Collaborative Administration: The Management of Successful Networks.” \textit{Public Management Review} 19 (3): 275–283. doi:10.1080/14719037.2016.1209236.
\item Dejarlais, R. (1994). \textit{Shelter blues. Sanity and selfhood among the homeless}. Philadelphia: University of Pennsylvania Press.
\item Delany-Moretlwe, S., Cowan, F. M., Busza, J., Bolton-Moore, C., Kelley, K., \& Fairlie, L. (2015). Providing comprehensive health services for young key populations: Needs, barriers and gaps. \textit{Journal of the International AIDS Society} 18, 19833.
\item Denis, J. L., Langley, A., \& Rouleau, L. (2007). Strategizing in pluralistic contexts: Rethinking theoretical frames. \textit{Human Relations}, 60(1), 179-215.
\item Denis, J. L., Lamothe, L., \& Langley, A. (2001). The dynamics of collective leadership and strategic change in pluralistic organisations. \textit{Academy of Management Journal}, 44(4), 809-837.
\item Dohan, D., Schmidt, L, \& Henderson, S. (2005). From enabling to bootstrapping: welfare workers’ views of substance abuse and welfare reform. \textit{Contemporary Drug Problems} 32 (3), 429-55. \url{https://doi.org/10.1177/009145090503200306}.
\item Durkheim, E. (1933). \textit{The Division of Labour in Society}. New York: Free Press.
\item Forbess, A., \& James, D. (2014). Acts of assistance. \textit{Social Analysis} 58 (3), 73-89. \url{https://doi.org/10.3167/sa.2014.580306}.
\item Furlotte, C., Schwartz, K., Koornstra, J. J., \& Naster, R. (2012). “Got a room for me?” Housing experiences of older adults living with HIV/AIDS in Ottawa. \textit{Canadian Journal on Aging/La Revue Canadienne Du Vieillissement} 31 (1), 37-48.
\item Furnari, S. (2014). Interstitial spaces: Microinteraction settings and the genesis of new practices between institutional fields. \textit{Academy of Management Review}, 39(4), 439-462.
\item Gherardi, S. (2009). Practice? It’s a matter of taste! \textit{Management Learning}, 40(5), 535-550.
\item Graham, M. (2004). Empowerment revisited: social work, resistance and agency in black communities. \textit{European Journal of Social Work}, 7(1), 43-56.
\item Grenier, A., Barken, R., Sussman, T., Rothwell, D. W., \& Bourgeois-Guérin, V. (2016). Homelessness among older people: Assessing strategies and frameworks across Canada. \textit{Canadian Review of Social Policy/Revue Canadienne de Politique Sociale}. 
\item Hupe, P., \& Buffat, A. (2014). A public service gap: Capturing contexts in a comparative approach of street-level bureaucracy. \textit{Public Management Review} 16 (4), 548-69. \url{https://doi.org/10.1080/14719037.2013.854401}.
\item Kemp, S. P., Marcenko, M. O., Hoagwood, K., \& Vesneski, W. (2009). Engaging parents in child welfare services: Bridging family needs and child welfare mandates. \textit{Child welfare}, 88(1), 101-126. 
\item Latour, B. (2005). \textit{Reassembling the social: An introduction to actor-network-theory}. Oxford: Oxford University Press.
\item Luhmann, Niklas (2002/1997) \textit{Die Gesellschaft der Gesellschaft}. Frankfurt am Mein: Darmstadt.
\item Mosse, D. (2004). Is good policy unimplementable? Reflections on the ethnography of aid policy and practice. \textit{Development and Change} 35(4), 636-671.
\item Mumby, D. K. (2005). Theorizing resistance in organization studies: A dialectical approach. \textit{Management Communication Quarterly}, 19(1), 19-44.
\item Nader, L. (1972). Up the anthropologist - perspectives gained from studying up. \textit{In Reinventing Anthropology}, ed. D. Hymes, New York: Pantheon, 283-311.
\item Nicolini, D. (2016). Is small the only beautiful? Making sense of ‘large phenomena’from a practice-based perspective. In \textit{The Nexus of Practices} (pp. 110-125). Routledge.
\item Putnam, L. L., Fairhurst, G. T., \& Banghart, S. (2016). Contradictions, dialectics, and paradoxes in organisations: A constitutive approach. \textit{The Academy of Management Annals}, 10(1), 65-171.
\item Taylor, J. R. (2007). Toward a theory of imbrication and organizational communication. \textit{The American Journal of Semiotics}, 17(2), 269-297.
\item Taylor, J. R. (2011). Organization as an (imbricated) configuring of transactions. \textit{Organization Studies}, 32(9), 1273-1294.
\item Walsh, C. A., Hewson, J., Paul, K., Gulbrandsen, C., \& Dooley, D. (2015). \textit{Falling through the cracks: Exploring the subsidized housing needs of low-income preseniors from the perspectives of housing provider}. SAGE Open 5 (3), 1-9.
\item Wagenaar, H. (2012). Dwellers on the threshold of practice: the interpretivism of Bevir and Rhodes. \textit{Critical policy studies}, 6(1), 85-99.
\item Watson, T. J. (2011). Ethnography, reality, and truth: The vital need for studies of ‘how things work’in organizations and management. \textit{Journal of Management Studies}, 48(1), 202-217.

%##################################################################################################################################################################################################################################
\end{thebibliography}