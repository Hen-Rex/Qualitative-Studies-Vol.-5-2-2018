%%%%%%%%%%%%%%%%%%%%%%%%%%%%%%%%%%%%%%%%%%%%%%%%%%%%%%%%
%%%%%%%%%%%%%%%%% START 1_ARTICLE.tex %%%%%%%%%%%%%%%%%%
%%%%%%%%%%%%%%%%%%%%%%%%%%%%%%%%%%%%%%%%%%%%%%%%%%%%%%%%

%\chapter{Introduction}
\lettrine[lines=2]{\bfseries\color{black}T}{he film} \textit{I, Daniel Blake}, directed by Ken Loach, portrays the everyday struggles of an elderly man navigating the English welfare system. Recovering from a heart attack, Blake experiences the bleak reality of an indifferent administration that leaves him hungry and homeless, while still making him jump through ever more absurd bureaucratic hoops. Largely seen as a commentary on current political developments in the United Kingdom under the Cameron and May governments, the film also addresses the quixotic reality of accessing welfare support in social systems around the world, as well as the absurd effects of bureaucracy in general.
\par
Beyond more overtly political questions about the extent and sufficiency of welfare provisions, a key question for welfare providers, recipients and academics alike has been how welfare services are  provided and received. Welfare states tend to have grown in a rather piecemeal historical process, including the development of various state, privatised or re-nationalised agencies, with multiple and sometimes overlapping obligations. As a result, it can be increasingly difficult to navigate between these organisations – both for service users and employees themselves. Indeed, public sector service has often been considered a matter of organising ‘pluralistic contexts’ and contents (Denis et al., 2001: 2007). In a complex public administration system, this often means that the service user is assessed by different agencies simultaneously, and the assessment or treatment of a user in one type of welfare service is likely to impact or impede claims in other service domains.   As a result, there is a growing scholarly interest in the interfaces within and between different welfare services and cross-boundary activities (Cristofoli et al., 2017). For instance, Forbess and James (2014) show how interstices emerge at the fringes of the public sector and in the tangle of public sector agents, businesses and civil society. Indeed, a lack of coherence or difficulty in connecting different organisational units is a widespread matter of concern where particularly vulnerable citizens are in danger of not receiving the support they need, of ‘falling through the cracks’ (Bartfeld, 2003; Walsh et al., 2015). This includes, but is not limited to, children (Bartfeld, 2003; Delany-Moretlwe et al., 2015; Kemp et al. 2009), elderly people (Burns, 2009; Furlotte et al., 2012; Grenier et al., 2016) and people with substance use issues (Delany-Moretlwe et al., 2015; Dohan, Schmidt, and Henderson, 2005; Forbess and James, 2014). These issues are well known, and in more general organisation studies literature the role of interstitial spaces or the betwixt-and-between spaces of formal organisations are recognised as critical for accomplishing organisational tasks (Kellogg, 2009; Furnari, 2014; Mumby, 2005), and yet, the way users may  fall short in these spaces and the new trajectories that are re-negotiated or stabilised, are often black-boxed. \par
A better understanding of the work that takes place at these interfaces of public administration requires an exploration of how welfare services are organised at a local level, from a qualitative stance but without reducing the accomplishment to a closed local activity. In the title of this introductory article, we open this work at the interfaces by using the terms imbrications and interstices as analytical concepts to capture the phenomena that are investigated in the articles of this special issue. The concept of imbrication in geology is a way to explain patterns of layering or sedimentation; imbrication means to be layered or ‘to be tiled’ and can be said to reflect a ‘history of successive accumulations’ (Taylor, 2007: 6). An analogy, thus, could be the ways in which roof tiles - if placed in the right way- overlap one another, so that rain does not come into the house. Organisationally speaking, imbrication suggests how ‘the practice’ of one local professional or administrator, ‘becomes the object of a different practice’ (Taylor 2011: 1285) - activities are layered. Being layered is well known in organising and welfare services; admittedly, imbrications can both make life difficult and pave the way for fairly smooth user trajectories and professional collaboration. Organising does not only happen in imbricated spaces, however, but also in spaces devoid of formal or layered structuring. The concept of interstices refers to gaps or breaks in something, such as sunshine filtered through a fence or a river running through a narrow gorge. In welfare systems or public sector organisation these interstices or breaks may be created through silos or divisional bureaucratic structures, national borders, the physical environment, dispersed geographical locations, finances or management accounting, different professional knowledges, and so on. Paraphrasing Furnari (2014: 440), we may say that the interstitial spaces of public sector organisation refers to ‘the small-scale settings’ where individuals (e.g. citizens, patients, professionals) from different fields or professions ‘interact occasionally and informally around common activities’ (Ibid.). In other words, the interstitial spaces highlight these gaps, or the fringes of formal organising; these spaces may be said to be situated betwixt-and-between formal regulation and have to be guided by the discretion of the professional or the capacities of the user and their ability to fill these gaps. By applying these concepts as analytical lenses, we investigate the local organisation and accomplishment of welfare services as a reciprocal movement between imbricated activities and interstitial spaces.
\par
In this Special Issue of \textit{\theJournal}, we bring together various theoretical and empirical approaches that tackle aspects of imbrication and interstices within and between public organisations. We wish to shed light on the way professionals and users organise at the interface of public sector organisations in the midst of different kinds of services, tools, policies, professional backgrounds and so on. This resonates with public debates within and about public services, where there is an increasing appetite for knowledge of, and possible solutions to, problems of the coordination of services, for interdisciplinary collaboration and for “holistic” views and strategies for helping users. We wish, through this Special Issue, to explore different thematic perspectives in order to provide a platform for bridging conversations across both countries and traditional scholarly boundaries. Much research on welfare services and implementation has pointed to the inadequacy of welfare organisations in helping users, and in allowing professionals to navigate the systems. This can be identified in all articles of this issue, but at the same time, the articles also ask what bridges the gaps, and how complex welfare systems come to work. Importantly, many scholars note that it remains important to disentangle the “black box” of the workings of public bureaucracy (cf. Mosse 2004, Bjerge \& Bjerregaard 2017, Bjerge \& Rowe 2017). We specifically advance this agenda by exploring ‘how things work’ in public administration (Watson, 2011) and by looking into the methodological issues in theorising these phenomena from a qualitative and practice-based stance.\pagebreak

\chapter{Accomplishing Public Sector Service: Theorising Interstices and Imbrications from a Qualitative Stance}
Crucially, many qualitative researchers of welfare service organisations have struggled to provide knowledge of issues of overlaps and gaps that go beyond the organisations they have studied. This focus on the many contexts of producing public sector service and the trenchant inter-organisational accomplishment resonates well with the focus in current practice-based organisation studies and how organising takes place in a nexus of practices. As Nicolini (2017: 102) notes, performances: ‘can be understood only if we take into account the nexus in which they come into being. What happens here and now and why (the conditions of possibility of any scene of action) is inextricably linked to what is happening in another ‘here and now’ or what has happened in another ‘here and now’ in the past’. In other words, the accomplishment of public sector services relies on ‘bundles of practices’ taking place elsewhere, and on the material-discursive framing of the situation (Nicolini, 2012). Practice-based approaches are much more prevalent in organisation studies (see e.g. Gherardi, 2009) compared to the literature on public administration (see e.g. Wagenaar, 2012), yet the focus on ‘doing’ and how work is accomplished in practice by the means of discursive and material elements is also an emerging agenda in public administration (ibid.). What the practice-based agenda encourages us to do is to explore the way that public administration actually comes to work, and to consider questions of space and the local ‘tactics’ of civil servants (de Certeau, 1984: Schatzki, 2018).\pagenote{Due to limitations in space, this special issue mainly focuses on the perspective of professionals dealing with interstitial spaces across welfare sectors and nation-states, and less on the perspective of citizens navigating the ‘welfare jungle’ (for this, see e.g. Bjerge and Nielsen, 2014; Graham, 2004; Dejarlais, 1994).} As such, practice-based studies of public administration resemble and advance the ambition of administrative ethnography to explore ‘what happens in practice when civil servants perform their daily work’ (Boll and Rhodes, 2015) and to expand the context of public services. That is, practice-based studies take a productive view on the mundane practices of public administration (Bjerge \& Bjerregaard, 2017), and thereby add knowledge to more “classic” views on public administration (Putnam et al., 2016), where models of and for organisational set-ups, as well as the actions of those inhabiting these organisations, tend to be well-defined, ordered and predictable. Accordingly, sufficient attention is not given to the fact that the workings of public administration, including its various internal and external boundaries, are continually constructed and reconstructed in variable relationships within everyday practice.
\par
The articles collected in this issue all approach public sector services with a practice-based sensitivity where topics such as advice giving, organising across professional boundaries, the tactics of the employees, casework, and local health policy are all opened up and expanded in new qualitative ways. As case studies, the articles do not give definite answers, but rather provide different pathways into the intersections or imbrications and interstitial spaces of the welfare services. With the authors´ shared interests in qualitative methods and an interpretative approach, the issue emphasises the contribution of the qualitative social science studies to addressing core societal debates about the welfare services, the work lives of its employees and the experiences of its users. The articles explore how far such an approach can take us in terms of understanding the everyday, mundane workings of contemporary welfare services, as well as how such studies tap into, elaborate or even challenge scholarly discussions of how to understand the phenomenon on a broader scale.  By taking their points of departure in various empirical settings, the articles explore the complexity of the services. Despite their dissimilar empirical starting points, they also address common features, which may be ascribed to the nature of welfare services per se.

\chapter{Contributions in This Special Issue}
\section{A Case Study of Casework Tinkering}
The first article in the Special Issue addresses the challenges of welfare provision in the Danish administration context, using the example of a single person. Like Daniel Blake, the drug user ‘Marianne’ struggles to navigate the welfare state, as her issues relate to multiple health and welfare service providers. Contrary to the film, however, Maj Nygaard-Christensen, Bagga Bjerge and Jeppe Oute present a compelling picture of the inner workings of public administration by analysing how staff from multiple agencies struggle to work together to advance Marianne’s case. Their article is based on extensive qualitative fieldwork on the interrelationship of drug treatment, psychiatric and employment services in Denmark and focuses on a temporary housing and drop-in facility called Oasis. The authors draw on human-technology relations literature and introduce the concept of ‘tinkering’ to the public administration debate. In this way they highlight the process of piecing together a suitable and sustainable treatment, and of the handling of Marianne’s case. This process is defined by continuous toing and froing between the different professionals who “quibble” over both practical measures and the interpretations of the issues at hand. The authors suggest that even though it is experienced as a strenuous procedure, ‘tinkering’ might be the key to successful welfare application, particularly in complex cases. This in turn calls for closer attention to the intricacies of the welfare state, and the way groups and individuals make policies work.

\section{Health Care Professionalism Without Doctors: Spatial Surroundings and Counter-Identification in Local Health Houses}
Drawing on a study of health care professionals in local “health houses” [da: Sundhedshuse] in two Danish municipalities, Østergaard Møller analyses how health houses comprise an alternative to the medical view of health. The article advances the ‘vignette method’ as a probing device for qualitative research, and scrutinises the processes through which the professionals constantly try to differentiate their services from what is offered in hospitals. The author includes an analysis of how institutional, and particularly spatial settings, are key for understanding such processes.  Originally, the health houses were established to pave the way for a “smoother” relationship between citizens and health professionals, and in terms of delivering health promotion and rehabilitation outside hospital - that is, to make health services more accessible for citizens and to make sure to imbricate between different types of health services. Østergaard Møller nevertheless demonstrates how these health houses and their particular spatial settings have also offered new opportunities for health professionals to redefine themselves as “health consultants” rather than, for example, as nurses. As the author argues, spatial surroundings that matter for professionalisation are often omitted, and as the article shows, spatial surroundings are used by the health consultants to identify and counter-identify with more established professions such as the medical profession. In other words, the health consultants and local health houses in many ways find their professional identity through the interstices of more established modes of professionalism in the public sector. Although health professionals and many of the citizens using the services are positive about the organisational set up of this new professional service, the article demonstrates how the spatial surroundings also create an ambiguous space where there is a lack of clarity about expectations and the role of the health professionals. This interstitial position creates a new demand for health consultants to fill in the gaps in their relationships with citizens.

\section{Using Ignorance as (\emph{Un})Conscious Bureaucratic Strategy: Street-Level Practices and Structural Influences in the Field of Migration Enforcement}
In this article, Lisa Marie Borrelli explores the case of street-level bureaucrats working in the field of migration enforcement. She explores the uneasy task of finding ir-regularised migrants and processing their cases – often until deportation. As the encounters are unforeseeable and characterised by tension and emotions, bureaucrats develop practices and strategies which help them to manage the often very personal encounters. While research has stressed the importance of ‘coping’ mechanisms and the problem of many ‘dirty’ hands, the author explores how ignorance is exploited as a tactic in the daily work of bureaucrats. In turn, she looks at how ignorance, including deliberate not-knowing, as well as non-deliberate partial-knowing or being kept ignorant, is used in public administration, through multi-sited, ethnographic fieldwork in migration offices and the border police/guard offices of three Schengen Member States: Sweden, Switzerland and Latvia. Borelli makes the distinction between structural and individual ignorance, which both have the ability to limit a migrant’s agency. By analyzing their intertwined relationship, Borrelli advances an understanding of how uncertainty and a lack of accountability become the results of everyday bureaucratic encounters. Ignorance, she argues, thus obscures state practices, subjecting migrants with precarious legal status to structural violence. In turn, Borelli´s findings underscore how interstices can be fabricated or enhanced by bureaucratic (in)-action.

\section{Assembling Advice}
This article by Samuel Kirwan, Morag McDermont and John Clarke explores the key role that voluntary sector advice agencies play for many citizens in the UK in accessing and understanding public services. These advice agencies may be conceived as fiercely ‘independent’ yet conflicted at the interstices of the welfare state. Presenting data from participant observation, interviews and focus groups with advisers and managers within the Citizens Advice Service, the paper explores this relationship by focusing on two particular areas of the service; the delivery of the service by volunteers, and the different funding streams that enable the service to function. The paper draws upon assemblage theory, focusing upon elements of an organisation in their ongoing practices and relationships; a processual approach that allows them to reflect upon the broader implications of their ethnographic data. The study discuss the ‘fragile futures’ of advice in the context of aggressive budget cuts and the welfare reform agenda, and shows how funding also becomes ‘a component of the advice assemblage with distinct effects upon the ongoing practice of advice´. The authors show how the local act of giving advice is a composition or assemblage of various components that encapsulates various sites, pasts and futures: the article demonstrates how the volontary advice organisations in this specific context seem necessary to help citizens navigate the welfare service. In underlining that, the article also points to the fragile future of these organisations, which thus potentially increases people’s danger of falling through the interstices of the welfare services. By focusing on advice assemblages, the article also notes the difficulty of navigating imbricated practices of funding.

\section{Treatment of Dual Diagnosis in Denmark --- Models for Cooperation and Positions of Power}
Katrine Schepelern Johansen takes her point of departure in her long-term experience of working as an anthropologist in the psychiatric system, focusing on patients who suffer from both a mental disorder and a substance use disorder – called dual diagnosis. These diagnoses are treated in two different organisational units or systems, a medically based psychiatric treatment system and a socially-oriented substance use treatment system. Despite the fact that a large proportion of the people in drug treatment, and of mental health patients, have these dual diagnoses, there is a remarkable lack of coordination and collaboration between the two organisational units. Applying David Brown´s (1983) classic concept of “organisational interfaces”, Johansen analyses the historical, political, organisational, professional and technical factors influencing the field of dual diagnosis treatment, and tests how far the concept of organisational interfaces can take us in understanding the challenges in the field. Not far enough, she concludes. A host of initiatives have been set up to enhance collaboration between the two systems. Some initiatives seem to have been able to generate equal collaboration between the two types of systems, however, in practice many initiatives can be characterised by a skewed balance between the two systems - predominantly with psychiatry being the most powerful and least interested in collaborating. The two systems and their employees do not enter into cooperation on equal terms and do not seem to agree on its necessity, as Brown´s model would have it. This means that treatment for dual diagnosis in the Danish welfare state is not at all coherent despite the fact that research repeatedly suggests it to be absolutely key for providing the proper help for persons with dual diagnosis. Johansen argues that these difficulties partially depend on unequal power relations at the organisational interfaces between psychiatric and substance use treatment.

\columnbreak %manual column break
\chapter{Cross-Pollinating Discussions and Contributions of the Special Issue}
Reading across the articles, it becomes evident that frictions in macro structures and policies are translated into challenges for organisations. We can thus trace the impact of gaps or interstices, and overlaps or imbrications, and a piecemeal welfare state by paying close attention to practices at the micro level. This includes negotiations within and between services, but also interactions with service users and their support networks. There is thus a value in ‘studying up’ on how welfare systems are not quite working smoothly (Nader, 1972). At the same time, the articles show that small-scale case studies need to be acutely aware of the external context of specific welfare provision, such as national policies, economic resources and professional traditions. To put it bluntly, no welfare supplier is or can be an island, and the connections and interrelations (or lack of) between agencies define their success and failure in terms of creating coherence for citizens in need of more than one type of services. As such, the activities performed by the welfare suppliers are inherently imbricated, and yet gaps or interstitial spaces are discernable. This stance encourages us to find ways of exploring the imbrications of these larger social phenomena, and not least to explore the large in the small (see also Latour, 2005). As case studies, the articles collected in this Special Issue offer both methodological and theoretical contributions to render the imbrications or interstices of public sector organisation visible.
\par
Methodologically, the articles collected in this issue move beyond individual case studies without negating the analytical depth necessary for meaningful qualitative research. Nygaard-Christensen, Bjerge and Oute show how a single person’s case can become a useful lens to show the conflicting interpretations and practical toolsets necessary to overcome organisational divides. Using a vignette method as a probing device, Østergaard Møller’s article highlights the spatial aspects of professionalism within the health sector, and provides a model to link the imbrication of places, discourses and labels. Borrelli’s work suggests that in tracing flows of (mis)information through organisational networks, we can improve our understanding of intentional and unintentional misapplications of policies. Kirwan, McDermont and Clarke compare different types of relationships between voluntary organisations and the British welfare state, and are thus able to highlight the uneasy relationship between funding struggles and voluntary work. Finally, Schepelern Johansen uses a kind of ‘meta auto-ethnography’ to highlight the continuity of organisational frictions across individual projects and sites, and offers new pathways of meaningful comparison between small-scale case studies.
\par
The articles draw on a wide range of sources and concepts for their theory – from technology studies (Nygaard-Christensen, Bjerge and Oute) to assemblage theory (Kirwan, McDermont and Clarke), or critically update classic public administration studies (Borrelli, Schepelern Johansen, Østergaard Møller). In this, they all provide interesting concepts to develop further. The article by Nygaard-Christensen, Bjerge and Oute shows how local case work also connects to configurations or categories made in practices elsewhere, and how these ‘narrow boxes’ or categories sometimes needs to be suspended or worked around – ‘tinkered with’ – in order to make things work. Organising case work involves navigating both overlapping or imbricated practices, but also interstitial spaces where the discretion of the professional is challenged in other ways. Østergaard Møller introduces a critical geography to the sociology of health care professions, and Schepelern Johansen shows the consistency of professional capital across different contexts and projects. Borrelli unpacks several forms of ignorance, both at a structural and an individual level. Her work, as well as that of Kirwan, McDermont and Clarke, discusses the ‘fragile futures’ of organisations and the impact of constant uncertainty and ambiguity. While they should not be seen to form a coherent theoretical approach, these themes – managing uncertainty, tinkering with cases and maintaining or overcoming professional divisions – are key issues that need to be explored in the context of complex welfare systems riddled with imbrications and interstices.
\par
In conclusion, we may say that problems at the interface of different kinds of work or services are not new problems.  The problems could also be related to old discussions of the consequences of the division of labour and increased functional differentiation, which has been part and parcel of sociological theory since Durkheim (1933) and Luhmann (1997). Similarly, the problem of welfare ‘gaps’ has been addressed in much of the public administration literature (Hupe and Buffat, 2014). As the articles collected in this issue show, however, these issues persist and do not seem to go away, even if particular efforts are made to ‘streamline’ and ‘integrate’ different service provisions. We believe that a practice-based and qualitative approach, as well as the concepts of imbrication and interstices, provide helpful devices for opening up work, problems and solutions at the interface of public sector organising. Interstitial spaces are always to some extent discernable; they emerge from architectural boundaries, professional differences and an overflow of formal organisation and so on. In one way these spaces have their advantages, for example in terms of ensuring citizens´ rights or clear cut boundaries in relation to who does what, yet, in another way, the interstitial spaces that we indicate are indeed also causing many problems, and require extremely hard work to stabilise. Similarly, imbrications are also always at stake, practices overlap, cultural norms fold into architecture, and may both enable each other but also at times constrain each other: figuratively speaking, the tiles may overlap more than needed. Academics thus need to keep studying these issues, and reminding policy makers that they persist. These ways of exploring the organisation of welfare services also have practical implications, since no service provider can be seen as successful if it fulfills its own goals, but in doing so collides with or hinders the broader goals of welfare provision and social equity.  Similarly, providing a service but not reaching key populations cannot be successful welfare delivery. Scholars thus need to be careful not to have a blinkered view in terms of the complexity of a given welfare system.