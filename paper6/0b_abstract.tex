    \begin{adjustwidth}{20mm}{20mm}
\label{paper6:abstract}
    \bigskip
    \begin{otherlanguage}{english}
    {\small
%    \fadebreak
%%%%%%%%%%%%%%%%%%%%%%%%%%%%%%%%%%%%%%%%%%%%%%%%%%%%%%%%
%%%%%%%%%%%%%%%% START ABSTRACT.tex %%%%%%%%%%%%%%%%%%%%
%%%%%%%%%%%%%%%%%%%%%%%%%%%%%%%%%%%%%%%%%%%%%%%%%%%%%%%%

\noindent Treatment for dual diagnosis in Denmark is divided between a medically based psychiatric treatment system and a socially oriented substance use treatment system; consequently, in order to deliver the most effective treatment to people with dual diagnosis, the two need to cooperate. A number of projects have been initiated to try out different models for cooperation, yet, on a larger, societal scale, we have not solved the puzzle of how it can be made to work in practice. My focus in this article is to suggest some reasons why it is so difficult to introduce cooperation between psychiatry and addiction treatment despite the many projects directed explicitly towards this. I suggest that at least part of the answer lies in the unequal power relations between psychiatry and substance use treatment.

\smallskip
\noindent\rule{\linewidth}{1pt}

%%%%% KEYWORDS
%
\noindent{\bfseries Keywords:}\hspace*{0.75em}{% list keywords below
dual diagnosis;
cooperation;
organizational interfaces;
power}.








    } % ends font family

%     \fadebreak

    \end{otherlanguage}

    \end{adjustwidth}